% !TEX root = main.tex
\section{First lecture: complex and Kähler manifolds}
We first review some complex geometry.
\begin{definition}
	A \tbf{complex manifold} is a locally ringed space $(X,\cO_X)$ such that
	\begin{itemize}
		\item $X$ is Hausdorff and second countable (this part is to ensure we actually have a topological manifold);
		\item $(X,\cO_X)$ is locally isomorphic to $(\Delta,\cO_\Delta)$, where $\Delta\subset \C^n$ is the polydisc.\qedbarhere 
	\end{itemize} 
\end{definition}
\begin{example}\label{eg: affine var}
	Let $f_i,\dots,f_d\in\C[z_1,\dots,z_n]$ be complex polynomials such that the Jacobian of
	\[f=(f_1,\dots,f_d):\C^n\to \C^d
	\]
	has everywhere full rank on the vanishing set $V=V(f)\subset \C^n$. By the holomorphic implicit function theorem (regular value theorem), $V$ is a complex manifold.
\end{example}
\begin{example}\label{eg: smooth algebraic var}
	If $X$ is a smooth algebraic variety over $\C$, we may cover it by affines $V_i$ which are of the same form as in \autoref{eg: affine var}. We may consider the analytic topology $X^{an}$ obtained by gluing the different charts $V_i$. Similarly, we may define the sheaf $\cO_X^{an}$ on $X^{an}$ by considering the sheaf of holomorphic functions on each $V_i$ (the transitions $V_i\to V_j$ are regular algebraic, hence holomorphic, so that this gluing makes sense). Then, $(X^{an},\cO_X^{an})$ is a complex manifold.
\end{example}

Note that in \autoref{eg: smooth algebraic var}, we obtain a natural map of ringed spaces
%
\[
\alpha:(X^{an},\cO_X^{an})\to (X,\cO_X)
\]
since the analytic topology is finer than the Zariski topology, and regular functions are holomorphic. In particular, we obtain a functor between abelian categories:
\[
\alpha^*:\cO_X\text{-mod}\to \cO_X^{an}\text{-mod}
\]
restricting to
\[
\alpha^*:\text{Coh}(X)\to \text{Coh}(X^{an}).
\]
\begin{theorem}[Géométrie algébrique géométrie analytique;\;\cite{Serre1956}]
	If $X$ is smooth\footnote{This can be dropped by considering complex analytic spaces (rather than manifolds).} and proper functor $\alpha^*:\text{Coh}(X)\to\text{Coh}(X^{an})$ is an equivalence, which moreover induces an isomorphism of sheaf cohomology.
\end{theorem}
\subsection{Almost complex structures}
A complex manifold $(X,\cO_X)$ has an underlying smooth manifold $(X, C^\infty_X)$, where $C^\infty_X$ denotes the sheaf of smooth functions on $X$; indeed, if $X$ has complex charts $U_i\subset \C^n$, the transitions are holomorphic, hence $C^\infty$. 
\begin{notation}
	Since the indices $i$ will be ubiquitous, $\iota$ shall denote the root of $-1$ for these notes (this spares the cumbersome $\sqrt{-1}$ alternative).
\end{notation}
On each chart $U_i$, we have multiplication by $\iota$, but this does not globalise, as $\iota$ does not commute with holomorphic functions: in the Taylor expansion, we have terms which are of degree $m$ where $m\neq 1\mod 4$. However, the differential of $\iota$ may be globalised, as we get rid of the higher order terms. In a local chart $U_i\subset \C^n$, the (real) tangent bundle has a local frame
\[
T_\R U_i=\langle \partial_{x_j},\partial_{y_j}:1\leq j\leq n\rangle,
\]
on which $I:=d\iota$ acts by
\[
\begin{cases}
	\partial_{x_j}\mapsto \partial_{y_j}\\
	\partial_{y_j}\mapsto -\partial_{x_j}.
\end{cases}
\]
\begin{definition}
	An \tbf{almost complex structure} on a smooth manifold $X$ is an endomorphism $I\in \text{End}(T_\R X)$ such that $I^2=-1$. We say that $I$ is integrable if $X$ is a complex manifold and $I$ is obtained by locally differentiating $\iota$. 
\end{definition}
\begin{question}\label{q: integrable?}
	Given $I$ an almost complex structure, when is it integrable?
\end{question}
Let us first set up some tools in order to address this question appropriately. Assume only for now that $X$ is a smooth manifold and $I$ is an almost complex structure. We can consider the complexified tangent bundle 
\[
T_\C X:=T_\R X\otimes_\R \C,
\]
to which we can extend the action of $I$. Since $I^2={-1}$, the minimal polynomial of $I$ is $x^2+1$, which is separable over $\C$, meaning that $I$ is diagonalisable, with eigenvalues $\pm \iota$. The eigenspaces must have the same dimension as $I$ acts on the \emph{real} tangent space. We thus obtain a decomposition
\[
T_\C X=T^{1,0}X\oplus T^{0,1}X= T^{1,0}X\oplus \overline{T^{1,0}X}.
\]
Note that we have
\begin{align*}
	T^{1,0}X=\{(v-iIv):v\in T_\C X\}&&T^{0,1}X=\{(v+iIv):v\in T_\C X\}.
\end{align*}
\begin{notation}
	We will use the following notation
	\begin{itemize}
		\item $\cA^0(X):=C^\infty_X$;
		\item $\cA^k(X)$ denotes the sheaf of (smooth) degree $k$ real forms;
		\item $\cA^k(X,\C)=\oplus_{p+q=k}\cA^{p,q}(X)$ denotes the sheaf of sections of $\bigwedge^kT_\C^*X$ (i.e. smooth complex degree $k$ forms) and $\cA^{p,q}(X)$ denotes the sheaf of sections of $\bigwedge^pT^{1,0}X\otimes \bigwedge^q T^{0,1}X$;
		\item $d:\cA^k(X,\C)\to \cA^{k+1}(X,\C)$ denotes the complexification of the usual exterior derivative, and can be decomposed by types as $d=\partial+\bar\partial$, where $\partial$ denotes the part corresponding to the differentiation in holomorphic coordinates, and similarly $\bar\partial$ for anti-holomorphic coordinates.
		\item $A^k(X)$, $A^k(X,\C)$, and $A^{p,q}(X)$ denotes the global sections of $\cA^k(X),\cA^k(X,\C)$, and $\cA^{p,q}(X)$ respectively.\
		\item $\cT_X$ denotes the sheaf of homolorphic vector fields, i.e. $\cT_X:=\cD er(\cO_X,\cO_X)$;
		\item $\Omega_X:=\cT_X^*$ denotes the cotangent sheaf.\qedbarhere
	\end{itemize}
\end{notation}
The following theorem answers \autoref{q: integrable?}.
\begin{theorem}[Newlander-Niremberg]
	$I$ is integrable if and only if $\bar{\partial}^2=0$.
\end{theorem}
Note that this is equivalent to $T^{1,0}X$ being closed under the (complex) Lie bracket (this is much related to the Frobenius theorem of differential geometry), and also equivalent of the vanishing of a certain tensor $N_I$ called the \emph{Nijenhuis} tensor. 
\subsection{Metrics}
Let $E$ be a real vector bundle on $(X,C_X^\infty)$. A \tbf{Riemannian metric} $g$ on $E$ is a section of $\text{Sym}^2E^\vee$ such that for all $p\in X$, $g_p$ is positive definite. If $E$ is a complex bundle, a \tbf{Hermitian metric} $h$ is a map of sheaves $E\otimes \overline E\to C^\infty(X,\C)$ such that each $h_p$ is Hermitian, i.e. $h_p(e,f)=\overline{h_p(f,e)}$ and $h_p(e,e)>0$ for all $e,f\in E_p$.  When $E=T_\R X$, we say that $g$ (resp. $h$) is a \tbf{Riemannian} (resp. \tbf{Hermitiam}) \tbf{metric on } $X$ (here, the almost complex structure $I$ is used to put a $\C$-structure on $T_\R X$).

If $X$ is a complex manifold and $h$ is a  Hermitian metric, then we can write
\[
h=g-i\omega
\]
where $g=\fR\fe(h)$ and $\omega=-\fI\fm(h)$. We obtain that $g$ is a Riemannian metric, and $\omega$ is skew-symmetric since
\[
\omega(X,Y)=\frac{\iota}{2}(h-\overline h)
\]
and $h$ is conjugate skew-symmetric. Thus, $\omega\in A^2(X)$.
\begin{definition}
	$(X,h)$ is \tbf{Kähler} if $d\omega=0$.
\end{definition}
That $h$ is linear in the first variable and anti-linear in the second ensures that $h(I-,I-)=h(-,-)$, implying that $g(I-,I-)=g(-,-)$, a property that is sometimes called \tbf{compatibility} of the metric with $I$. We have 
\[
\omega(-,-)=\frac{\iota}{2}(h(-,-)-\overline h(-,-))=\frac{1}{2}(h(I-,-)+\overline h (I-,-))=g(I-,-),
\]
which also implies
\[
\omega(-,I-)=g(-,-).
\]
\begin{definition}\label{def: kahler}
	A form $\omega\in A^2(X)$ is called \tbf{positive} if $\omega(u,Iu)>0$ for all $u\in T_\R X$. We see that a de Rham cohomology class in $H^2(X,\C)$ is \tbf{positive} if it can be represented by a positive form. If moreover $\omega$ is $I$-invariant (or equivalently, of type $(1,1)$ after embedding $A^2(X)\subset A(X,\C)$), we say $\omega$ is \tbf{Kähler}.
\end{definition}
If $\omega$ is Kähler, we may define the hermitian metric $h_\omega=\omega(-,I-)-i\omega$, and we have that $\omega$ is Kähler if and only if $(X,h_\omega)$ is Kähler.
\begin{example}
	Let $X=\bP^n$, with projective coordinates $Z_0,\dots,Z_n$. Let $U_i$ be the $Z_i\neq 0$ chart, and define $z_j=\frac{Z_j}{Z_i}$. We may define on $U_i$ the metric
	\[
	\omega_{FS}=\omega=i\partial\overline\partial\log\pa{1+\sum_jz_j\overline z_j},
	\]
	and one checks that these glue to a global form, which we call the \tbf{Fubini-Study metric}. Written as a Kähler potential this way shows that it is a Kähler metric.
\end{example}
Note that if $(X,\omega)$ is Kähler, restricting the metric to a complex submanifold $Y$ preserves all properties of \autoref{def: kahler}, and so $(Y,\omega_Y)$ is Kähler. Thus, any projective manifold is Kähler.

\subsection{Connections}

Let $E$ be a complex (the real case is identical) vector bundle on $(X,C_X^\infty)$. A \tbf{complex connection} in $E$ is a $\C$-linear map 
\[
\nabla:\cA^0(E,\C)\to \cA^1(E,\C),
\]
(here $\cA^i(E,\C)= \cA^i(X,\C)\otimes \Gamma(E)$) such that
%
\[
\nabla(f\cdot s)=df\otimes s+f\cdot\nabla s
\]
for all section $s$ of $E$ and $f\in C^\infty_X$.

If $E$ is a holomorphic bundle on a complex manifold, we can define the operator
%
\[
\overline\partial:\cA^0(E)\to \cA^{0,1}(E)
\]
%
as follows: if $\sigma_i$ is a local frame, and $s=s^i \sigma_i$ a section, we let
\begin{equation}
	\overline\partial(s^i\sigma_i):=(\overline\partial s^i)\otimes \sigma_i.\label{eq: dbar bundle}
\end{equation}

Indeed, given another frame $\tau_j$ related by $\sigma_i=g_{ij}\tau_j$, we have
\[
\overline\partial (s^i)\otimes \sigma_i=\overline \partial(s_i)\otimes g_{ij}\tau_j=\overline\partial(g_{ij}s^i)\otimes \tau_j
\]
since the transitions $g_{ij}$ are holomorphic by assumption.

A (complex) connection being valued in $\cA^1(E,\C)=\cA^{1,0}(E)\oplus\cA^{0,1}(E)$, we may split $\nabla=\nabla^{1,0}+\nabla^{0,1}$.
\begin{definition}
	The complex connection $\nabla$ in $E$ is said to be \tbf{compatible} with the holomorphic structure if $\nabla^{0,1}=\overline\partial$. Suppose $E$ has a hermitian metric $h$. We say $\nabla$ is \tbf{compatible} with $h$ if for any sections $e,f$, we have equality of forms
	\[d(h(e,f))=h(\nabla e,f)+h(e,\nabla f).\]
	More geometrically, this says that $h$ is parallel to the connection, i.e. constant along parallel transport, i.e. the connection has $U(n)$-holonomy. We say $\nabla$ is a \tbf{Chern connection} if it is both compatible with the holomorphic structure and the hermitian metric.
\end{definition} 

\begin{theorem}[Chern]
	There exists a unique Chern connection.
\end{theorem}

When $E=T_\R X$, the Chern connection ought to be regarded as the complex geometric analogue of the Levi-Civita connection from Riemannian geometry. In fact this is more than an analogy. If $h$ is a hermitian metric, the Levi-Civita connection of $g=\fR\fe(h)$ can be complexified to a complex connection. It is a theorem that the Levi-Civita connection is the Chern connection if and only if $(X,h)$ is Kähler.

We can extend the connection $\nabla:\cA^0(E,\C)\to \cA^1(E,\C)$ to a connection
\[
\nabla:\cA^p(E,\C)\to \cA^{p+1}(E,\C)
\]
for all positive $p$ via
\[
\nabla(\omega\otimes s)=d\omega\otimes s+(-1)^p\omega\wedge\nabla s,
\]
where $\omega$ is a $p$-form and $s$ is a section of $E$.

\begin{remark}
	Note that this is different from the usual extension of a connection to tensors since we are dealing with skew-symmetric forms. In particular, this satisfies a different Leibniz rule:
	\[
	\nabla(fs)=df\wedge s\otimes d\nabla s.
	\]
\end{remark}

\begin{definition}
	We define the \tbf{curvature} of $\nabla$ to be the composition $\nabla^2=\nabla\circ\nabla=F_\nabla$.
\end{definition}

Note that
\begin{align*}
	\nabla(\nabla fs)=\nabla(df\otimes s+f\nabla s)&=ddf-df\wedge\nabla s+\nabla(f\nabla s)	\\
	&=-df\wedge\nabla s+df\wedge\nabla s+f\nabla^2s=f\nabla^2s
\end{align*}
so that $F_\nabla$ is $C^\infty_X$-linear, that is a section of $\cA^2(End(E),\C)$.

We may also define
%
\[F_\nabla^k:=\underbrace{F_\nabla\circ\cdots\circ F_\nabla}_k\in \cA^{2k}(End(E),\C).
\]
We define the \tbf{$k$th Chern character} of $\nabla$ to be
%
\[
\text{ch}_k(E,\nabla):= \Tr\pa{\frac{1}{k!}\pa{\frac{\iota}{2\pi}F_\nabla^k}}\in A^{2k}(X,\C).
\]

\begin{theorem}[Chern-Weil]
	The following is true about the Chern character.
	\begin{enumerate}
		\item $\text{ch}_k(E,\nabla)$ is closed;
		\item The cohomology class $\text{ch}_k(E):=[\text{ch}_k(E,\nabla)]\in H^{2k}_{dR}(X,\C)$ is independent of $\nabla$;
		\item $\text{ch}_k(E)$ is real, i.e. in $H^{2k}_{dR}(X,\R)$ (in fact, it is integral);
		\item The total Chern character $\sum_k\text{ch}_k(E)$ is equal to the cohomology class of $\Tr(\exp(\frac{\iota}{2\pi}F_\nabla))$ (this one directly follows from developing the exponential).
	\end{enumerate}	
\end{theorem}
