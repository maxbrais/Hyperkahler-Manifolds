% !TEX root = main.tex
\section{Fifth Lecture}


\subsection{The Hilbert square of a K3 surface} Let $S$ be a K3 surface. We will construct a new manifold from $S$, called the Hilbert scheme of two points, denoted $S^{[2]}$.
This is the first example of a hyperkähler manifold of dimension more than $2$ that we shall see.


% Construction via blow-up and quotient (Notes p.1-2)
We start with the product $S^2:=S\times S$. Let $\Delta\subset S^2$ be the diagonal. We consider the blow-up $\tilde{S^{2}}:=\text{Bl}_\Delta(S^{2})$.

\begin{remark}[Blow-ups in the holomorphic category]
	This is a good place to recall how to deal with blow-ups in the holomorphic category.
	Let $Z$ be a subvariety regularly embedded\footnote{This means locally as many equations as codimension. If for example $Z$ is smooth, this condition is always satisfied.} in $X$ which is defined by an ideal sheaf $\cI$. By definition of a regular embedding, we can locally on an open set $U\subset X$, find generators $\cI=\langle f_1,\dots,f_k\rangle$. We define a map
	\begin{align*}
		\underline{f}:U\setminus (U\cap Z)&\to \bP^{k-1}\\
		x&\mapsto [f_1(x):\cdots:f_k(x)].
	\end{align*}
	The blow-up $\text{Bl}_{Z\cap U}(U)$ is defined as the closure of the graph $\Gamma(\underline{f})$ inside $U\times \bP^{k-1}$.  This picture is local, but naturally glues to the blow-up $\text{Bl}_{Z}(X)$ of $X$ at $Z$.
	It satisfies the same universal property as in the algebraic case: any map to $X$ whose pre-image of $Z$ is a divisor\footnote{If $X$ is not smooth, one must replace the word ``divisor'' by ``Cartier divisor''.} factors through the blow-up. 
\end{remark}

Let $\iota:S^2\to S^2$ be the involution $\iota(x,y)=(y,x)$. The \tbf{symmetric product} is the quotient $S^{(2)}:=S^2/\iota$. 

\begin{remark}\label{rem:sing-sym2}
	The quotient $S^{(2)}$ is singular along the image of the diagonal since the action has stabilisers on the diagonal, which has codimension more than $1$. Indeed, locally around the diagonal, we can use the dimension of the fixed locus to see that $S^{(2)}$ is of the form
	\[
	\C^2\times\C^2/\pm. 
	\]
	Now, $\C^2/\pm =\Spec\C[x^2,xy,y^2]=\Spec\frac{\C[u,w,z]}{uw-z^2}$, so that the singularities are of the form of those of a cone times $\C^2$. Since blowing up the cone at the origin resolves the singularities, similarly, blowing up $S^{(2)}$ at the image of the diagonal resolves all singularities. 
\end{remark}
We define the \tbf{Hilbert Square} of $S$ to be the blow-up $S^{[2]}\xrightarrow{h}S^{(2)}$ at the image of the diagonal, which  is smooth by \autoref{rem:sing-sym2}. The map $h$ is an example of the so-called \emph{Hilbert--Chow} morphism, i.e. the map $\text{Hilb}^n(X)\to \text{Sym}^nX$. It is a theorem by Fogarty (see \cite{Fogarty1968}) that the Hilbert scheme of $n$ points of a surface is smooth, and so that the Hilbert--Chow morphism is a resolution of singularities in those instances.
\begin{remark}
	We can also define the Hilbert scheme of $n$ points of $X$ in a functorial way: a map $T\to X^{[n]}$ corresponds to a flat map
	\[
	X\times T\supset Y\to T,
	\]
	whose fibres $Y_t$ are all zero-dimensional schemes of length $n$ (i.e. embedded deformations of $n$ points in $X$).
\end{remark}

\begin{remark}
	The involution $i$ lifts to an involution $\tilde{i}$ on the blow-up $\tilde{S^2}$ of $S^2$ at the diagonal, and a careful analysis of the local form of this action and of the singularities of $S^{(2)}$ shows that $S^{[2]}$ is also the quotient of $\tilde S^2$ by $\tilde i$. In particular, we have a Cartesian diagram
	\[
	\begin{tikzcd}
		\tilde{S^2}\arrow[loop left, "\tilde i"] \arrow[d, "\pi"'] \arrow[r, "q"] & S^{[2]} \arrow[d, "h"] \\
		S^2\ar[loop left, "i"] \arrow[r, "p"]                               & S^{(2)}
	\end{tikzcd}
	\]
\end{remark}

Let $E\subset S^{[2]}$ be the exceptional divisor of $h$. The map $q$ is a double cover branched along $E$. In particular, we have a divisor class $\delta$ such that $\cO(\delta)$ is the Tschirnhausen line bundle of this cover. This implies that we have $2\delta =E$.
%
\begin{proposition}\label{prop: S[2] HK}
	Let $S$ be a K3 surface. Then $S^{[2]}$ is a 4-dimensional hyperkähler manifold. Moreover, we have an isomorphism
	\[
	H^2(S^{[2]},\Z)\cong H^2(S,\Z)\oplus \Z\delta,
	\]
	compatible witth the Hodge structure.
\end{proposition}
\begin{remark}
Taking $S^{[n]}$ also yields a hyperkähler manifold of dimension $2n$. However, it is false that given any hyperkähler manifold $X$, its Hilbert scheme of $n$-points is also Hyperkähler.
\end{remark}

We sketch the proof of \autoref{prop: S[2] HK}, starting with the fundamental group.

\begin{lemma}\label{lem:pi1-surj}
 The map $q$ induces a surjection $q_*:\pi_1(\tilde{S^2})\to \pi_1(S^{[2]})$. 
\end{lemma}
\begin{proof}
	This is a property of ramified double covers and more generally of ramified covers such that the preimage of some point in the branched locus is set theoretically one point.
	
	We can compute the fundamental group by taking loops based at some point $p\in E$. Given any such path $\gamma:[0,1]\to S^{[2]}$, we can deform it so that only $\gamma(0)=\gamma(1)=p$ are in $E$. Since $\tilde S^2\setminus q^{-1}(E)\to S^{[2]}\setminus E$ is étale, we can lift the path $\gamma:(0,1)\to S^{[2]}\setminus E$ to a path $\tilde \gamma: (0,1)\to \tilde S^2\setminus q^{-1}(E)$, and by continuity, the limits $\tilde\gamma(0)$ and $\tilde \gamma(1)$ must lie in the (set-theoretical) preimage of $p$. But since $q$ is a double cover, this set-theoretical image is only one point, and  so $\gamma$ lifts to a path $\tilde \gamma$.
	\end{proof}
	We shall also make use of the following fact.
	\begin{lemma}\label{lem: pi1 codimension}
		Let $X$ be a real manifold and $Z\subset X$ a subvariety of (real) codimension $c$. The map $\pi_1(X\setminus Z)\to \pi_1(X)$ induced by inclusion is surjective if $c\geq 1$ and an isomorphism if $c\geq 2$.
	\end{lemma}
	\begin{proof}
		If $c\geq 1$, we can move loops to avoid $Z$, and if $c\geq 2$, we can also move homotopies between those loops to avoid $Z$.
	\end{proof}
	
	\begin{proposition}
		$S^{[2]}$ is simply connected.
	\end{proposition}
	\begin{proof}
		Since $S$ is simply connected, $S^2$ is also simply connected. Let $\Delta\subset S^2$ be the diagonal. By \autoref{lem: pi1 codimension}, $S^2\setminus \Delta\simeq \tilde S^2\setminus q^{-1}(\Delta)$ is also simply connected, and by \autoref{lem: pi1 codimension} again, $\tilde S^2$ is simply connected (the exceptional divisor has real codimension $2$). Thus, by \autoref{lem:pi1-surj}, we conclude that $S^{[2]}$ is simply connected.
	\end{proof}
	\begin{lemma}
		$S^{[2]}$ is Kähler.
	\end{lemma}
	\begin{proof}
		$S^2$ is Kähler, and the blow-up of $\tilde S^2$ is again Kähler; see \cite[Proposition 3.24]{Voisin1998}.  Let $\omega$ be a Kähler form on $\tilde S^2$. We claim the form $q_*\omega$ is Kähler. Indeed, by a criterion of Demailly and Paun (see \cite{DemaillyPaun2004}), to show that $q_*\omega$ is Kähler, it suffices to show that for all analytic subvariety $Y\subset S^{[2]}$, we have
		\[
		\int_Y q_*\omega^{\dim Y}=\int_{S^{[2]}} q_*\omega^{\dim Y}\cup [Y]>0,
		\]
		and that $-\omega$ is not Kähler.\footnote{This second condition is only present to avoid pathologies with complex manifolds that have no odd-dimensional complex subvarieties.} But by the projection formula, we have that
		\[
		\int_{S^{[2]}} q_*\omega^{\dim Y}\cup [Y]=\int_{\tilde S^2}\omega^{\dim{f^{-1}(Y)}}\cup [f^{-1}(Y)]=\int_{f^{-1}(Y)}\omega^{\dim(f^{-1}(Y))}>0
		\]
		since $\omega$ is Kähler. Since $S^{[2]}$ has a divisor $E$, it is clear that $-\omega$ is not Kähler, as
		\[
		\int_{E}(-\omega)^3=-\int_E\omega^3<0
		\]
		by what we have shown.
 	\end{proof}
 	\begin{proposition}
 		$S^{[2]}$ has trivial canonical bundle
 	\end{proposition}
 	\begin{proof}
 		 Since $K_{S^2}$ is trivial, we have
 		\[
 		K_{\tilde S^2}=\cO(\tilde E).
 		\]
 		Similarly, using Hurwitz formula, we have
 		\[
 		K_{\tilde S^2}=\cO(E)\otimes q^*K_{S^{[2]}},
 		\]
 		so that $q^*K_{S^{[2]}}=\cO_{\tilde S^2}$. Pushing forward, we see that $K_{S^{[2]}}$ is $2$-torsion, and using that $q$ is a cyclic covering, we have moreover an isomorphism
 		\[
 		K_{S^{[2]}}\oplus K_{S^{[2]}}(-\delta)\simeq \cO_{S^{[2]}}\oplus \cO(-\delta).
 		\]
 		This isomorphism must be given by a matrix of the type
 		\[
 		\begin{pmatrix}
 			H^0(K_{S^{[2]}}^\vee)& H^0(K^\vee_{S^{[2]}}(\delta))\\H^0(K_{S^{[2]}}^\vee(-\delta))&H^0(K_{S^{[2]}}^\vee).
 		\end{pmatrix}
 		\]
 		Now, if $K_{S^{[2]}}\not\simeq\cO_{S^{[2]}}$, we have $H^0(K_{S^{[2]}}^\vee)=0$ since $K_{S^{[2]}}=K_{S^{[2]}}^\vee$. This implies we must have an isomorphism $K^\vee_{S^{[2]}}(-\delta)\simeq \cO_{S^{[2]}}$, i.e. that $K_{S^{[2]}}\simeq \cO(\delta)$, and so taking the square, $\cO_{S^{[2]}}\simeq \cO(E)$, which is impossible since $E$ is an effective divisor. Thus, we must have $K_{S^{[2]}}\simeq \cO_{S^{[2]}}$.
 	\end{proof}
	We are now ready to prove that $S^{[2]}$ is hyperkähler.
\begin{proof}[Proof of \autoref{prop: S[2] HK}]
	By the Künneth formula, we have that 
	\[
	H^2(S^2,\Q)=p_1^*H^2(S,\Q)\oplus p_2^*H^2(S,\Q)\simeq H^2(S,\Q)\oplus H^2(S,\Q),
	\]
	where $p_i:S^2\to S$ are the projections. The involution $i ^*$ acts on this space by switching both summands, and so we see that
	\[
	H^2(S^2,\Q)^{\mu_2}=\{p_1^*\alpha+p_2^*\alpha:\alpha\in H^2(S,\Q)\},
	\]
	and so we have an isomorphism
	\begin{align}
		H^2(S,\Q)&\xrightarrow{\sim}H^2(S^2,\Q)^{\mu_2}\label{eq:inv-coh}\\
		\alpha&\mapsto p_1^*\alpha+p_2^*\alpha=:\tilde \alpha.\notag
	\end{align}
  We now recall some facts from topology. First, 
 	over $\Q$ the cohomology of the quotient by a free action is given exactly by the invariant cohomology classes. Therefore, we have an isomorphism
 	\[
 	H^2(S^{[2]}\setminus E,\Q)=H^2(S^{(2)}\setminus p(\Delta),\Q)=H^2(S^2\setminus \Delta,\Q)^{\mu_2}.
 	\]
 	Now, from the long exact sequence in relative cohomology, and the identification $H^k(S^2,S^2\setminus \Delta;\Q)=H^{k-4}(\Delta,\Q)$ (since $\Delta$ has codiemension $4$) we have
 	\[
 	\cdots\to H^{-2}(\Delta,\Q)\to H^2(S^2,\Q)\to H^2(S^2\setminus\Delta,\Q)\to H^{-1}(\Delta,\Q)\to \cdots
 	\]
 	Thus, the left and right terms vanish, and so pullback gives an isomorphism $H^2(S^2,\Q)\simeq H^2(S^2\setminus \Delta,\Q)$, and in particular,
 	\[
 	H^2(S,\Q)\simeq H^2(S^2,\Q)^{\mu_2}\simeq H^2(S^2\setminus\Delta,\Q)^{\mu_2}\simeq H^2(S^{[2]}\setminus E,\Q).
 	\]
 	The exact same argument also shows that 
 	\[
 	H^2(S^{(2)},\Q)\simeq H^2(S^{(2)}\setminus p(\Delta),\Q)=H^2(S^2\setminus\Delta,\Q)^{\mu_2}=H^2(S^2,\Q)^{\mu_2}.
 	\]
 	%
 	We now use the long exact sequence in relative cohomology for the pair $(S^{[2]},S^{[2]}\setminus E)$:
 	\[
 	\cdots\to H^1(S^{[2]}\setminus E,\Q)\to H^0(E,\Q)\to H^2(S^{[2]},\Q)\to H^2(S^{[2]}\setminus E,\Q)\to H^1(E,\Q)\to \cdots
 	\]
 	Recall that $E$ is the projectivisation of the normal bundle of $\Delta\subset S\times S$, i.e. the projectivisation of the tangent bundle $\cT_S$. We have a formula for the cohomology of such projectivisation:
 	\[
 	H^\bullet(E,\Q)=H^\bullet(\bP(\cT_S),\Q)=\frac{H^\bullet(S,\Q)[\zeta ]}{(\xi^2+c_1(\cT_S)\zeta+c_2(\cT_S))},
 	\]
 	where $\zeta$ is the first Chern class of $\cO_{\bP(\cT_X)}(1)$, and so has degree $2$. Since $S$ is simply connected, we see from the formula that $H^1(E,\Q)=0$.  Moreover, by \autoref{lem: pi1 codimension}, $S^{[2]}\setminus E$ is simply connected, and therefore $H^1(S^{[2]}\setminus E,\Q)=0$ by the universal coefficient theorem. Thus, we have a short exact sequence
 	\begin{equation}
 	0\to H^0(E,\Q)\to H^2(S^{[2]},\Q)\to H^2(S^{[2]}\setminus E,\Q)\to 0\label{eq:ses-exc},
 	\end{equation}
 	Or equivalently by what we shown,
 	\[
 	0\to H^0(E,\Q)\to H^2(S^{[2]},\Q)\to H^2(S^{(2)},\Q)\to 0.
 	\]
 	Since we are over $\Q$, this sequence splits, ad we have
 	\[
 	H^2(S^{[2]},\Q)=H^2(S^{(2)},\Q)\oplus [E]\Q= H^2(S,\Q)\oplus \delta\Q.
 	\]
 	Now suppose we have an integral form
 	\[
 	\omega\in H^2(S^{[2]},\Z).
 	\] 
 	Since $S^{[2]}$ is simply connected, $H^2(S^{[2]},\Z)$ has no torsion. We can therefore split $\omega=a\tilde \alpha + b\delta$ where $\alpha\in H^2(S,\Z)$ is assumed not to be divisible and $a,b\in \Q$. Let $s\in S$ and consider the rational line $l$ lying over $s\in S\simeq p(\Delta)\subset S^{(2)} $ in the exceptional divisor. Here, we must recall $\tilde E\simeq E$ and that $\tilde E=\bP(\cN_{\Delta/S^2})=\bP(\cT_S) $ as $\tilde S^2\to S^2$ is a smooth blow-up. Note that 
 	\[
 	\int_l\tilde \alpha=0
 	\]
 	since $l$ is a fibre and $\tilde \alpha$ is pulledback from $H^2(S^{(2)},\Z)$. On the other hand, we have
 	%
 	\[
 	2\int_l \delta=\int_{f^{-1}(l)}\tilde E.
 	\]
 	But $f^{-1}(l)=2l'$, where $l'$ is the reduced pre-image of $l$ along $q$, and so
 	\[
 	\int_l\delta=\int_{l'}\tilde E.
 	\] 
 	Since $\tilde S^2\to S^2$ is a blow-up, we have $\cO(\tilde E)|_{\tilde E}=\cO_{\bP(\cT_S)}(-1)$ (this generalises the well-known case of the blow-up of a point and is seen directly by looking at the Rees algebra), and so we have $\cO(\tilde E)|_{l'}=\cO_{l'}(-1)$, and thus
 	\[
 	\int_l\delta=-1.
 	\]
 	Therefore, we have
 	\[
 	\int_l\omega=a\int_l\tilde\alpha+b\int_l\delta=-b.
 	\]
 	Since $[l]$ is an integral cochomogy class, we conclude that $b\in\Z$. This implies that $b\delta$ is integral, and so that $a\tilde \alpha$ is integral. Now, for any $\gamma\in H^6(S^{(2)},\Z)$, we have
 	\[
 	\int_{S^{(2)}}a\tilde\alpha\cup\gamma=\int_{S^{[2]}}a\tilde\alpha\cup q^*\gamma
 	\]
 	since $h$ is birational. Since $q^*\gamma$ is integral, this quantity is an integer. Since the pairing is unimodular and $\tilde \alpha$ is not divisible by assumption, we may pick $\gamma$ so that 
 	\[
 	\int_{S^{(2)}}\tilde\alpha\cup\gamma=1,
 	\]
 	so that $a\in \Z$. Thus, we have shown that
  	
 	\begin{equation}
 	H^2(S^{[2]},\Z)=\delta\Z\oplus H^2(S,\Z).\label{eq:decomp-h2}
 	\end{equation}
 	From \eqref{eq:inv-coh}, the map $H^2(S,\Z)\to H^2(S^2,\Z)^{\mu_2}=H^2(S,\Z)$ is induced by sums and pullback, and so respects the Hodge structure. Moreover, $\delta$ is a pure $(1,1)$-class, and so \eqref{eq:decomp-h2} respects the Hodge structure.
 	 
 	There remains only to show that $H^0(S^{[2]},\Omega_{S^{[2]}}^2)$ is spanned by a symplectic form. Since \eqref{eq:decomp-h2} respects the Hodge decomposition, we know $H^{0}(S^{[2]},\Omega_{S^{[2]}}^2)\simeq H^0(S,\Omega_S^2)$ is one dimensional, and so it suffices to show that there exists a holomoprhic symplectic form on $S^{[2]}$. Let $\sigma$ be a holomorphic symplectic form on $S$. Then, we know there exists a holomorphic $2$-form $\tilde \sigma$ on $S^{[2]}$ whose pullback to $S^2\setminus \Delta$ is of the form
 	\[
 	p_1^*\sigma+p_2^*\sigma.
 	\]
 	Now, $\tilde \sigma$ is symplectic if and only if $\tilde \sigma^2\in H^0(S^{[2]},\Omega_{S^{[2]}}^4)$ is a non vanishing section. Since $\Omega_{S^{[2]}}^4=K_{S^{[2]}}=\cO_{S^{[2]}}$, it suffices to show that $\tilde \sigma^2$ does not vanish at a single point, i.e. that $\tilde \sigma$ is symplectic at one point. Thus, we may check this away from $E$, and since $S^2\setminus \Delta\to S^{[2]}\setminus E$ is étale, it suffices to check this for the form $p_1^*\sigma+p_2^*\sigma$ on $S^2\setminus \Delta$, which is trivial (the direct sum of two symplectic forms is symplectic).
\end{proof}

\subsection{Local systems}

We introduce local systems and holonomy, important concepts in differential geometry.

\begin{definition}
	Let $B$ be a connected manifold and let $A$ be a ring (e.g., $\Z,\Q,\R,\C$). A \tbf{local system} of $A$-modules on $B$ is a sheaf $\bH$ of $A$-modules which is locally isomorphic to a constant sheaf with fiber $V$, where $V$ is an $A$-module.
\end{definition}

Given a local system $\bfH$ (over $A=\R$ or $\C$), we can associate a vector bundle
\[
\cH:=\bfH\otimes_A \cA^0(B,A),
\]
where $\cA^0(B)$ is the sheaf of smooth functions. This bundle has a canonical flat connection
\[
\begin{tikzcd}
	\cH\ar[d,equal]\ar[r,"\nabla"]& \cH\otimes_{\cA^0(B,A)}\cA^1(B,A)\ar[d,equal]\\
	\bfH\otimes_A \cA^0(B,A)\ar[r]&\bfH\otimes_A \cA^1(B,A)
\end{tikzcd}
\]
defined by $\nabla:=\id_\bfH\otimes_A d$. Indeed, this is a connection: we can write locally a section $s\in\cH$ as 
\[
s=e_i\otimes s^i
\]
where the elements $e_i$ form a basis of $V$, and we have
\[
\nabla(f\cdot s)=e_i\otimes_A d(fs^i)=e_i\otimes_A (fds ^i+(df)s^i)=df \otimes_{\cA^{0}(B,A)} s+f\nabla s
\]

 This connection is flat; indeed, we have
 \[
 \nabla(\nabla(e_i\otimes s^i))=\nabla(e_i\otimes ds^i)=e_i\otimes dds^i+ds^i\wedge \nabla e_i=0
 \]

% Equivalence between local systems and flat connections (Notes p.7-8)
\begin{theorem}
	There is a bijective correspondence between (real or complex) local systems of rank $r$ on $B$ and pairs $(\cE,\nabla)$ where $\cE$ is a (real of complex) vector bundle of rank $r$ on $B$ and $\nabla$ is a flat connection.
\end{theorem}
\begin{proof}[Sketch of proof (see {\cite[Chapitre 9]{Voisin1998}})]
	We have already shown that a local system yields a vector bundle and a flat connection. Conversely, one may show that given a vector bundle and a flat connection, the sheaf of parallel sections yields a local system.
\end{proof}
%
Let $(E,\nabla)$ be a vector bundle with a connection on some manifold $X$. Let $\gamma:[0,1]\to X$ be a path from $x=\gamma(0)$ to $y=\gamma(1)$. The pullback bundle $(\gamma^*E,\gamma^*\nabla)$ on $[0,1]$ is flat for dimension reasons (recall that the curvature is a matrix of $E$-valued $2$-forms). Thus, $(\gamma^*E,\gamma^*\nabla)$ corresponds to a local system $H_\gamma$ on $[0,1]$. Since $[0,1]$ is contractible, any vector bundle is topologically trivial, and therefore this local system is trivial, which provides a canonical isomorphism $E_x\xrightarrow{\sim}E_y$ between the fibers.

\begin{definition}
	The isomorphism $E_x\xrightarrow{\sim} E_y$ is called the \tbf{parallel transport} along $\gamma$.
\end{definition}

If $x=y$, we consider the space of loops based at $x$, $\Omega(x)$. The parallel transport defines a map 
\[ 
\rho: \Omega(x)\to \text{Aut}(E_x).
\]

\begin{definition}
	The \tbf{holonomy group} $\text{Hol}(E,\nabla)\subseteq \text{GL}(E_x)$ of the connection $\nabla$ is the image of $\Omega(x)$ under $\rho$.
	The \tbf{reduced holonomy group} $\text{Hol}_0(E,\nabla)$ is the image of the subgroup $\Omega_0(x)$ of contractible loops.
\end{definition}

\begin{remark}
	If $X$ is connected, $\text{Hol}(E,\nabla)$ does not depend on the choice of $x$ up to isomorphism. Indeed, any path $x\mapsto y$ gives an isomorphism between the holonomy groups at $x$ and at $y$. Note however that this isomorphism is path-dependent. Nonetheless, if there is no reduced holonomy, then this identification is only dependent on paths up to homotopy. Moreover, note that if the connection $\nabla$ is flat, the reduced holonomy is trivial: indeed, any contractible loop lies in a contractible neighbourhood of $x$, which is in particular a neighbourhood where the local system is trivial. Conversely, it is not hard to see that if the reduced holonomy is trivial, then the connection is flat: on a contractible neighbourhood, every fibre is canonically isomorphic. Thus, the connection is flat if and only if the reduced holonomy is trivial. In this case, the morphism $\rho$ descends to a representation
	\[
	\overline \rho:\pi_1(X,x)\to \text{GL}(E_x)
	\]
	which is called the \tbf{monodromy} of the flat connection $\nabla$ (or of the associated local system).
\end{remark}

If $(X,g)$ is a Riemannian manifold, we consider the Levi-Civita connection $\nabla$ on the tangent bundle $T_\R X$. Recall that this connection is uniquely determined by two conditions
\begin{itemize}
	\item $\nabla g=0$ (i.e. $g$ is parallel, that is, the holonomy commutes with $g$)\footnote{Recall/learn that given a connection on a bundle $E$ (in our case the tangent bundle), we may extend the connection to a $(m,n)$-tensor $\alpha$ by declaring
	\begin{align*}
	(\nabla_Z\alpha)(X_1,\dots,X_m,\omega_1,\dots,\omega_m):=Z(\alpha(X_1,\dots,X_m,\omega_1,\dots,\omega_m))&-\sum_{i=1}^m\alpha(X_1,\dots,\nabla_ZX_i,\dots,X_m,\omega_1,\dots,\omega_n)\\&-\sum_{i=1}^n\alpha(X_1,\dots,X_n,\omega_1,\dots,\nabla_Z\omega_i,\dots,\omega_n),
	\end{align*}
 where $Z,X_1,\dots,X_m$ are vector fields and $\omega_1,\dots,\omega_n$ are forms. In our situation, we obtain that $(\nabla_Zg)(X_1,X_2)=Z(g(X_1,X_2))-g(\nabla_ZX_1,X_2)-g(X_1,\nabla_ZX_2)$. Thus, $\nabla g=0$ translates to $Z(g(X_1,X_2))=g(\nabla_ZX_1,X_2)+g(X_1,\nabla_ZX_2)$, and this condition is often called compatibiiity of $g$ with the connection.
}
	\item The connection has no torsion, i.e. $\nabla_XY-\nabla_YX=[X,Y]$, where the bracket is the Lie bracket.
\end{itemize}
An important theorem (see \cite[Theorem 4.17]{Ballmann2006}) states that a complex manifold is Kähler if and only if the (complexification of the) Levi-Civita connection coincides with the Chern connection.\footnote{One ought to be precise here: complexifying the Levi-Civita connection to $T_\C X$ and then restricting to $T^{1,0}X$ yields the Chern connection.} 

We will denote $\text{Hol}(X,g):=\text{Hol}(T_\R X, \nabla)$. A fundamental phenomenon of the theory of connections is the following.

\begin{proposition}[Holonomy Principle]
	Let $(X,g)$ be a Riemannian manifold. Then, the space of tensors parallel to the Levi-Civita connection
	\[
	A_{par}(T X^{\otimes a}\otimes T^*X^{\otimes b}):=\{\alpha\in\Gamma(TX^{\otimes a}\otimes T^*X^{\otimes b})\mid \nabla\alpha=0\}
	\]
	is equal to the space of tensors $\alpha$ such that for any $p\in U$, the restriction map
	\[
	A_{par}(TU^{\otimes a}\otimes T^*U^{\otimes b})\to T_pU^{\otimes a}\otimes T_p^*U^{\otimes b}
	\] 
	is an isomorphism onto the space of holonomy invariant tensors.
\end{proposition}

\begin{example}\label{eg: hol-CY}
	Since $g$ is by definition parallel to the Levi-Civita connection, the holonomy commutes with the metric. In other words, $\text{Hol}(X,g)\subset O(n)$. If $X$ is oriented, the natural orientation 
	\[
	\sqrt{\det g_{ij}}dx^1\wedge\cdots\wedge x^{n}
	\]
	can be computed to be parallel, and so $\text{Hol}_0(X,g)\subseteq SO(n)$.
	
	If $X$ is Kähler, one can show that the almost complex structure $I$ is parallel. This implies $Hol(X,g)\subseteq U(m)$, where $m=\dim_\C X$. It is an important theorem (see \cite[Theorem 4.17]{Ballmann2006}) that a Riemannian manifold is Kähler if and only if its holonomy lies in $U(m)$. In other words, Kähler manifolds are exactly those oriented Riemannian manifolds whose parallel transport is $\C$-linear.
	
	If $(X,\omega)$ is Kähler and Calabi--Yau of complex dimension $m$ in the stronger sense that $K_X=\cO_X$, then by definition, the tangent bundle is Ricci-flat (with respect to the Chern connection of some Kâhler metric, by the now proved Calabi conjecture). One can compute that this implies that the induced connection on $\det TM$, and dually on $K_X$, is flat. Since the canonical bundle is trivial, the corresponding local system is trivial. We have therefore a global parallel  $(m,0)$-form (i.e. a ``holomorphic orientation''), so that using the holonomy principle, we find that the holonomy lies in $SU(m)$.  
	
	Conversely, suppose a Riemannian manifold $X$ has global $SU(m)$ holonomy. One can compute that the parallel transport on the determinant $\det TX$ is given by the (complex) determinant of the parallel transport on $TX$, implying that there is no (global) holonomy. Thus, $\nabla_{\det TX}$ is flat and $\det TX$ moreover corresponds to the trivial local system, and so is topologically trivial. Taking a global parallel smooth $(m,0)$-form $\Omega$, and using that the determinant of the Chern connection is the Chern connection of the determinant, we find that $\nabla_{\det TX}^{0,1}=\op_{\det TX}$. By considering types, we must have $\nabla_{\det TX}^{1,0}\Omega=0$. Thus, $\op_{\det TX}\Omega=0$, i.e. $\Omega$ is holomorphic. This implies that $\det TX$ is holomorphically trivial, and so that $X$ is Calabi--Yau in the stronger sense.
	
	In some sense, this shows that Kähler manifolds are the complex analog of Riemannian manifolds, and that Calabi--Yau manifolds are the complex analog of oriented Riemannian manifolds. We will show that this analogy extends to hyperkähler manifolds, which are the complex analog of symplectic manifolds.
\end{example}
After this discussion on Calabi--Yau manifolds, it is a good point to introduce the Bochner principle.
\begin{proposition}[Bochner's Principle]
	If $(X,\omega)$ is a compact Kähler manifold with $\text{Ric}(\omega)=0$, then all holomorphic tensors are parallel.
\end{proposition}
\begin{remark}
	Note that by the (proven) Calabi--Yau conjecture, if $X$ is Calabi--Yau (i.e. $c_1(X)=0$), then it is possible to find a Kähler metric whose Ricci-curvature is trivial. Therefore, by Bochner's principle, it is always possible to find a Kähler metric on a Calabi--Yau manifold such that all holomorphic tensors are parallel to the corresponding Chern/Levi-Civita connection.
\end{remark}

We use this principle to give a purely Riemannian geometric characterisation of hyperkähler manifolds. Recall that the complex symplectig group $Sp(m)\subset U(2m)\subset SO(4m)$ is that of endomorphisms preserving a given \emph{complex} symplectic form.

\begin{theorem}\label{thm: HK holonomy}
	Let $(X,g)$ be a compact Riemannian manifold. Then $X$ is hyperkähler\footnote{Or rather can be endowed with a complex structure making it a hyperkähler manifold.} if and only if $\text{Hol}(X,g)=Sp(m)$.
\end{theorem}
\begin{proof}
	($\Leftarrow$) Assume that $\text{Hol}(X,g)=Sp(m)$. By definition, we can find a holonomy-invariant symplectic complex two form at a point $p\in X$. By the holonomy principle, this extends to a global parallel $(2,0)$-form $\sigma$ on $X$ since $X$ is simply connected. By the definition of the extension of the Levi-Civita connection to forms and using that the Levi-Civita connection is torsion-free, we have
	\[
	(d\sigma)(X_1,X_2,X_3)=(\nabla_{X_1}\sigma)(X_2,X_3)-(\nabla_{X_2}\sigma)(X_1,X_3)+(\nabla_{X_3}\sigma)(X_1,X_2)=0,
	\]
	so that $\sigma$ is closed. But using $d\sigma=\p\sigma+\op\sigma$ and comparing types, we must have $\op\sigma=0$, i.e. $\sigma\in H^0(X,\Omega_X^2)$ is holomorphic.
	
	We must verify that $H^0(X,\Omega_X^2)$ is spanned by $\sigma$. By Bochner's principle, any holomorphic $2$-form $\sigma'$ must be parallel, and using the holonomy principle, this implies that the space of holomorphic $2$-forms is contained in the space of $Sp(m)$ invariant (complex) symplectic forms at a point $p$. But this space is known to be of one (complex) dimension, so that $\sigma'=c\sigma$ for some constant $c$. Moreover, by \autoref{eg: hol-CY}, we know that $X$ is Kähler and $K_X=\cO_X$, so that we conclude $X$ is hyperkähler.
	
	($\Rightarrow$) Assume $X$ is hyperkähler. It admits a holomorphic symplectic form $\sigma$ which is parallel by Bochner's principle. By the Holonomy Principle, $\text{Hol}(X,g)$ preserves $\sigma$, so $\text{Hol}(X,g)\subseteq Sp(m)$. One can then use Berger's classification of holonomy groups to show that in fact we have the equality $\text{Hol}x(X,g)=Sp(m)$.
\end{proof}

\begin{corollary}
	Let $X$ be a hyperkähler manifold of dimension $2m$. Then
	\[
	H^0(X,\Omega_X^k)=\begin{cases}
		0 & \text{if } k \text{ is odd}\\
		\C\cdot \sigma^{k/2} & \text{if } k \text{ is even}.
	\end{cases}
	\]
	In particular, the holomorphic Euler characteristic is $\chi(X,\cO_X)=m+1$.
\end{corollary}
\begin{proof}
	By Bochner's principle, holomorphic forms are parallel. By the Holonomy Principle and \autoref{thm: HK holonomy}, they correspond to $Sp(m)$-invariant elements in $\bigwedge^k V^*$. It is a fact from representation theory that the space of $Sp(V,\sigma)$-invariants in $\bigwedge^k V^*$ is spanned by powers of $\sigma$ if $k$ is even, and is zero if $k$ is odd.
	
	The Euler characteristic is $\chi(X,\cO_X) = \sum_k (-1)^k h^k(X,\cO_X)$. By Hodge symmetry, $h^k(X,\cO_X)=h^{0,k}(X)=h^{k,0}(X)=h^0(X,\Omega_X^k)$, and so that $\chi(X,\cO_X)=m+1$ follows directly.
\end{proof}
%
Here is a table to remember the different holonomies of the different types of Riemannian manifolds we have discussed. For each row, the condition on holonomy can be given as a purely Riemannian definition of this type of manifold.\\ 
\[
\resizebox{\textwidth}{!}{%
	\begin{tabular}{|l|c|r|r|}
		\hline
		Manifold&Real dimension&Holonomy&Reason\\
		\hline
		&&&\\
		Riemannian& $m$ & contained in $O(m)$& preserves metric\\&&&\\
		Oriented Riemannian& $m$ & contained in $SO(m)$& preserves orientation\\&&&\\
		Symplectic & $2m$ & contained\footnotemark\;\;in $Sp(m,\R)\subset SO(2m)$& preserves the symplectic form\\
		&&&\\
		Kähler& $2m$& contained in $U(m)$& preserves the complex structure\\&&&\\
		Calabi--Yau $(K_X=\cO_X)$&$2m$&contained in $SU(m)$& preserves the ``complex orientation''\\
		&&&(top holomorphic form)\\&&&\\
		Hyperkähler & $4m$& equal to $Sp(m,\C)\subset SU(2m)$& preserves the holomorphic\\
		&&&symplectic form\\
		\hline
	\end{tabular}
}
\]
\footnotetext{Here, it must be said that contrary to the other manifolds in the list, the holonomy is taken with respect to a ``symplectic'' connection, which is in general not the same as the Levi-Civita connection.}
\\

These groups are part of Berger's classification of reduced holonomy groups.\footnote{The only other possible reduced holonomy groups of an irreducible Riemannian manifold  that is not locally a symmetric space are $Sp(n)U(1)$, $Spin(7)$ and $G_2$.} In fact, Berger proved that In the case of a Kähler manifold, the reduced holonomy has to be \emph{exactly}  $U(m),SU(m)$ or $Sp(m)$. In particular, a non hyperkähler Calabi--Yau manifold has reduced holonomy \emph{exactly} $SU(m)$, and a non Calabi--Yau Kähler manifold has reduced holonomy \emph{exactly} $U(m)$.

