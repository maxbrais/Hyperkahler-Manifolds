\documentclass[11pt]{amsart}




\newcommand{\cA}{\mathcal{A}}
\newcommand{\cB}{\mathcal{B}}
\newcommand{\cC}{\mathcal{C}}
\newcommand{\cD}{\mathcal{D}}
\newcommand{\cE}{\mathcal{E}}
\newcommand{\cF}{\mathcal{F}}
\newcommand{\cG}{\mathcal{G}}
\newcommand{\cH}{\mathcal{H}}
\newcommand{\cI}{\mathcal{I}}
\newcommand{\cJ}{\mathcal{J}}
\newcommand{\cK}{\mathcal{K}}
\newcommand{\cL}{\mathcal{L}}
\newcommand{\cM}{\mathcal{M}}
\newcommand{\cN}{\mathcal{N}}
\newcommand{\cO}{\mathcal{O}}
\newcommand{\cP}{\mathcal{P}}
\newcommand{\cQ}{\mathcal{Q}}
\newcommand{\cR}{\mathcal{R}}
\newcommand{\cS}{\mathcal{S}}
\newcommand{\cT}{\mathcal{T}}
\newcommand{\cU}{\mathcal{U}}
\newcommand{\cV}{\mathcal{V}}
\newcommand{\cW}{\mathcal{W}}
\newcommand{\cX}{\mathcal{X}}
\newcommand{\cY}{\mathcal{Y}}
\newcommand{\cZ}{\mathcal{Z}}

\newcommand{\fA}{\mathfrak{A}}
\newcommand{\fB}{\mathfrak{B}}
\newcommand{\fC}{\mathfrak{C}}
\newcommand{\fD}{\mathfrak{D}}
\newcommand{\fE}{\mathfrak{E}}
\newcommand{\fF}{\mathfrak{F}}
\newcommand{\fG}{\mathfrak{G}}
\newcommand{\fH}{\mathfrak{H}}
\newcommand{\fI}{\mathfrak{I}}
\newcommand{\fJ}{\mathfrak{J}}
\newcommand{\fK}{\mathfrak{K}}
\newcommand{\fL}{\mathfrak{L}}
\newcommand{\fM}{\mathfrak{M}}
\newcommand{\fN}{\mathfrak{N}}
\newcommand{\fO}{\mathfrak{O}}
\newcommand{\fP}{\mathfrak{P}}
\newcommand{\fQ}{\mathfrak{Q}}
\newcommand{\fR}{\mathfrak{R}}
\newcommand{\fS}{\mathfrak{S}}
\newcommand{\fT}{\mathfrak{T}}
\newcommand{\fU}{\mathfrak{U}}
\newcommand{\fV}{\mathfrak{V}}
\newcommand{\fW}{\mathfrak{W}}
\newcommand{\fX}{\mathfrak{X}}
\newcommand{\fY}{\mathfrak{Y}}
\newcommand{\fZ}{\mathfrak{Z}}

\newcommand{\fa}{\mathfrak{a}}
\newcommand{\fb}{\mathfrak{b}}
\newcommand{\fc}{\mathfrak{c}}
\newcommand{\fd}{\mathfrak{d}}
\newcommand{\fe}{\mathfrak{e}}
\newcommand{\ff}{\mathfrak{f}}
\newcommand{\fg}{\mathfrak{g}}
\newcommand{\fh}{\mathfrak{h}}
\newcommand{\fj}{\mathfrak{j}}
\newcommand{\fk}{\mathfrak{k}}
\newcommand{\fm}{\mathfrak{m}}
\newcommand{\fn}{\mathfrak{n}}
\newcommand{\fo}{\mathfrak{o}}
\newcommand{\fp}{\mathfrak{p}}
\newcommand{\fq}{\mathfrak{q}}
\newcommand{\fr}{\mathfrak{r}}
\newcommand{\fs}{\mathfrak{s}}
\newcommand{\ft}{\mathfrak{t}}
\newcommand{\fu}{\mathfrak{u}}
\newcommand{\fv}{\mathfrak{v}}
\newcommand{\fw}{\mathfrak{w}}
\newcommand{\fx}{\mathfrak{x}}
\newcommand{\fy}{\mathfrak{y}}
\newcommand{\fz}{\mathfrak{z}}

\newcommand{\bA}{\mathbb{A}}
\newcommand{\bB}{\mathbb{B}}
\newcommand{\bC}{\mathbb{C}}
\newcommand{\bD}{\mathbb{D}}
\newcommand{\bE}{\mathbb{E}}
\newcommand{\bF}{\mathbb{F}}
\newcommand{\bG}{\mathbb{G}}
\newcommand{\bH}{\mathbb{H}}
\newcommand{\bI}{\mathbb{I}}
\newcommand{\bJ}{\mathbb{J}}
\newcommand{\bK}{\mathbb{K}}
\newcommand{\bL}{\mathbb{L}}
\newcommand{\bM}{\mathbb{M}}
\newcommand{\bN}{\mathbb{N}}
\newcommand{\bO}{\mathbb{O}}
\newcommand{\bP}{\mathbb{P}}
\newcommand{\bQ}{\mathbb{Q}}
\newcommand{\bR}{\mathbb{R}}
\newcommand{\bS}{\mathbb{S}}
\newcommand{\bT}{\mathbb{T}}
\newcommand{\bU}{\mathbb{U}}
\newcommand{\bV}{\mathbb{V}}
\newcommand{\bW}{\mathbb{W}}
\newcommand{\bX}{\mathbb{X}}
\newcommand{\bY}{\mathbb{Y}}
\newcommand{\bZ}{\mathbb{Z}}

\newcommand{\ba}{\mathbb{a}}
\newcommand{\bb}{\mathbb{b}}
\newcommand{\bc}{\mathbb{c}}
\newcommand{\bd}{\mathbb{d}}
\newcommand{\be}{\mathbb{e}}
\newcommand{\bg}{\mathbb{g}}
\newcommand{\bh}{\mathbb{h}}
\newcommand{\bi}{\mathbb{i}}   
\newcommand{\bj}{\mathbb{j}}
\newcommand{\bk}{\mathbb{k}}   
\newcommand{\bl}{\mathbb{l}}
\newcommand{\bm}{\mathbb{m}}   
\newcommand{\bn}{\mathbb{n}}
\newcommand{\bo}{\mathbb{o}}
\newcommand{\bp}{\mathbb{p}}  
\newcommand{\bq}{\mathbb{q}}
\newcommand{\br}{\mathbb{r}}   
\newcommand{\bs}{\mathbb{s}}   
\newcommand{\bt}{\mathbb{t}}
\newcommand{\bu}{\mathbb{u}}
\newcommand{\bv}{\mathbb{v}}
\newcommand{\bw}{\mathbb{w}}
\newcommand{\bx}{\mathbb{x}} 
\newcommand{\by}{\mathbb{y}}
\newcommand{\bz}{\mathbb{z}}

\newcommand{\bfA}{\mathbf{A}}
\newcommand{\bfB}{\mathbf{B}}
\newcommand{\bfC}{\mathbf{C}}
\newcommand{\bfD}{\mathbf{D}}
\newcommand{\bfE}{\mathbf{E}}
\newcommand{\bfF}{\mathbf{F}}
\newcommand{\bfG}{\mathbf{G}}
\newcommand{\bfH}{\mathbf{H}}
\newcommand{\bfI}{\mathbf{I}}
\newcommand{\bfJ}{\mathbf{J}}
\newcommand{\bfK}{\mathbf{K}}
\newcommand{\bfL}{\mathbf{L}}
\newcommand{\bfM}{\mathbf{M}}
\newcommand{\bfN}{\mathbf{N}}
\newcommand{\bfO}{\mathbf{O}}
\newcommand{\bfP}{\mathbf{P}}
\newcommand{\bfQ}{\mathbf{Q}}
\newcommand{\bfR}{\mathbf{R}}
\newcommand{\bfS}{\mathbf{S}}
\newcommand{\bfT}{\mathbf{T}}
\newcommand{\bfU}{\mathbf{U}}
\newcommand{\bfV}{\mathbf{V}}
\newcommand{\bfW}{\mathbf{W}}
\newcommand{\bfX}{\mathbf{X}}
\newcommand{\bfY}{\mathbf{Y}}
\newcommand{\bfZ}{\mathbf{Z}}

\newcommand{\bfa}{\mathbf{a}}
\newcommand{\bfb}{\mathbf{b}}
\newcommand{\bfc}{\mathbf{c}}
\newcommand{\bfd}{\mathbf{d}}
\newcommand{\bfe}{\mathbf{e}}
\newcommand{\bff}{\mathbf{f}}
\newcommand{\bfg}{\mathbf{g}}
\newcommand{\bfh}{\mathbf{h}}
\newcommand{\bfi}{\mathbf{i}}
\newcommand{\bfj}{\mathbf{j}}
\newcommand{\bfk}{\mathbf{k}}
\newcommand{\bfl}{\mathbf{l}}
\newcommand{\bfm}{\mathbf{m}}
\newcommand{\bfn}{\mathbf{n}}
\newcommand{\bfo}{\mathbf{o}}
\newcommand{\bfp}{\mathbf{p}}
\newcommand{\bfq}{\mathbf{q}}
\newcommand{\bfr}{\mathbf{r}}
\newcommand{\bfs}{\mathbf{s}}
\newcommand{\bft}{\mathbf{t}}
\newcommand{\bfu}{\mathbf{u}}
\newcommand{\bfv}{\mathbf{v}}
\newcommand{\bfw}{\mathbf{w}}
\newcommand{\bfx}{\mathbf{x}}
\newcommand{\bfy}{\mathbf{y}}
\newcommand{\bfz}{\mathbf{z}}

\newcommand{\sA}{\mathsf{A}}
\newcommand{\sB}{\mathsf{B}}
\newcommand{\sC}{\mathsf{C}}
\newcommand{\sD}{\mathsf{D}}
\newcommand{\sE}{\mathsf{E}}
\newcommand{\sF}{\mathsf{F}}
\newcommand{\sG}{\mathsf{G}}
\newcommand{\sH}{\mathsf{H}}
\newcommand{\sI}{\mathsf{I}}
\newcommand{\sJ}{\mathsf{J}}
\newcommand{\sK}{\mathsf{K}}
\newcommand{\sL}{\mathsf{L}}
\newcommand{\sM}{\mathsf{M}}
\newcommand{\sN}{\mathsf{N}}
\newcommand{\sO}{\mathsf{O}}
\newcommand{\sP}{\mathsf{P}}
\newcommand{\sQ}{\mathsf{Q}}
\newcommand{\sR}{\mathsf{R}}
\newcommand{\sS}{\mathsf{S}}
\newcommand{\sT}{\mathsf{T}}
\newcommand{\sU}{\mathsf{U}}
\newcommand{\sV}{\mathsf{V}}
\newcommand{\sW}{\mathsf{W}}
\newcommand{\sX}{\mathsf{X}}
\newcommand{\sY}{\mathsf{Y}}
\newcommand{\sZ}{\mathsf{Z}}

\newcommand{\sa}{\mathsf{a}}
\newcommand{\sd}{\mathsf{d}}
\newcommand{\se}{\mathsf{e}}
\newcommand{\sg}{\mathsf{g}}
\newcommand{\si}{\mathsf{i}}
\newcommand{\sj}{\mathsf{j}}
\newcommand{\sk}{\mathsf{k}}
\newcommand{\sm}{\mathsf{m}}
\newcommand{\sn}{\mathsf{n}}
\newcommand{\so}{\mathsf{o}}
\newcommand{\sq}{\mathsf{q}}
\newcommand{\sr}{\mathsf{r}}
\newcommand{\st}{\mathsf{t}}
\newcommand{\su}{\mathsf{u}}
\newcommand{\sv}{\mathsf{v}}
\newcommand{\sw}{\mathsf{w}}
\newcommand{\sx}{\mathsf{x}}
\newcommand{\sy}{\mathsf{y}}
\newcommand{\sz}{\mathsf{z}}

\newcommand{\q}[1]{``#1''}
\newcommand{\tbf}[1]{\textbf{#1}}
\newcommand{\Specc}{\text{\Spec}}
\newcommand{\mf}{\mathfrak}
\newcommand{\maxinv}{\text{maxinv}}
\newcommand{\maxord}{\text{maxord}}
\newcommand{\ord}{\text{ord}}
\newcommand{\pnp}{\par\null\par}
\newcommand{\Aut}{\text{Aut}}
\newcommand{\Ext}{\text{Ext}}
\newcommand{\Z}{\mathbb Z}
\newcommand{\ra}{\rightarrow}
\newcommand{\la}{\leftarrow}
\newcommand{\modulo}{\text{ mod }}
\newcommand{\Tr}{\text{Tr}}
\newcommand{\sgn}{\text{sgn}}
\newcommand{\ddx}{\frac{\partial}{\partial x}}
\newcommand{\ddy}{\frac{\partial}{\partial y}}
\newcommand{\ddz}{\frac{\partial}{\partial z}}
\newcommand{\im}{\text{im }}
\newcommand{\Hom}{\text{Hom}}
\newcommand{\colim}{\text{colim}}
\newcommand{\nsubeq}{\text{ }\trianglelefteq\text{ }}
\newcommand{\nsub}{\text{ }\triangleleft\text{ }}
\newcommand{\lcm}{\text{lcm}}
\newcommand{\modd}{\text{  mod}}
\newcommand{\Spec}{\text{Spec }}
\newcommand{\supp}{\text{supp }}
\newcommand{\Q}{\bQ}
\newcommand{\R}{\mathbb R}
\newcommand{\C}{\mathbb C}
\newcommand{\RP}{\mathbb{RP}}
\newcommand{\CP}{\mathbb{CP}}
\newcommand{\mbf}{\mathbf}
\newcommand{\nn}[1]{\lVert#1\rVert}
\def\dashmapsto{\mathrel{\mapstochar\dashrightarrow}}
\newcommand{\pa}[1]{\left(#1\right)}
\newcommand{\brc}[1]{\left\{#1\right\}}
\newcommand{\brk}[1]{\!\left[#1\right]}
\newcommand{\id}{\text{id}}



\usepackage{hyperref}
\usepackage{graphicx}
\usepackage[utf8]{inputenc}
\usepackage[T1]{fontenc}
\usepackage{amsfonts}
\usepackage{amssymb}
\usepackage{amsthm}
\usepackage{xspace}
\usepackage{lplfitch}
\usepackage{csquotes}
\usepackage{faktor}
\usepackage{comment}
\usepackage{tikz-cd}
\usetikzlibrary{cd}
\usepackage{hyperref}
\usepackage{fullpage}
\usepackage{amssymb,amsmath}

\usepackage{fullpage}
\usepackage{url}
\usepackage{geometry}
\geometry{
a4paper,
total={180mm,257mm},
left=15mm,
top=15mm,
}
\usepackage{tikz}
\usetikzlibrary{shapes,arrows}
\usetikzlibrary{intersections}
\usepackage[backend=biber,style=alphabetic,url=true, doi=true,maxbibnames=99,maxcitenames=99,backref=true]{biblatex}\addbibresource{reference.bib}


\newif\ifqedbarused
\newcommand{\qedbar}{%
  \unskip\nobreak\hfill % Push content to the right
  \hspace{1em} % Add space before the bar (adjust "1em" as needed)
  \rule{3em}{0.4pt} % Horizontal bar
}
\newcommand{\qedbarhere}{%
  \qedbar%
  \global\qedbarusedtrue% Suppress automatic bar
}



%%%%%%%%%%%%%%%%%%%%%%%%%%%%%%%%%%%%%%%%%%%%%%%%%
% Margin notes
%%%%%%%%%%%%%%%%%%%%%%%%%%%%%%%%%%%%%%%%%%%%%%%%%
\usepackage{todonotes}
\newcommand\todoin[2][]{\todo[inline, caption={2do}, #1]{\begin{minipage}{\textwidth-4pt}#2\end{minipage}}}
\newcommand{\brent}[2][]{\todo[#1,color=blue!20!white,size=scriptsize]{\textbf{B:} #2}}
\newcommand{\maxim}[2][]{\todo[#1,color=red!20!white,size=scriptsize]{\textbf{M:} #2}}


%%%%%%%%%%%%%%%%%%%%%%%
% Colours & links
%%%%%%%%%%%%%%%%%%%%%%%

\usepackage{xcolor,hyperref}
\definecolor{darkred}{rgb}{.8,0,0}
\definecolor{tocolor}{rgb}{.1,.1,.1}
\definecolor{urlcolor}{rgb}{.2,.2,.6}
\definecolor{linkcolor}{rgb}{.1,.1,.5}
\definecolor{citecolor}{rgb}{.4,.2,.1}
\definecolor{gray}{rgb}{.8,.8,.8}

\hypersetup{
	backref=true,
	colorlinks=true,
	urlcolor=urlcolor,
	linkcolor=linkcolor,
	citecolor=citecolor,
	pdfauthor = {Maxim Brais},
	}
	
%\newcommand{\email}[1]{\href{mailto:#1}{#1}}

\newcommand{\thmautorefname}{Theorem}
\renewcommand{\sectionautorefname}{Section}
\renewcommand{\subsectionautorefname}{Section}
\renewcommand{\subsubsectionautorefname}{Section}

%%%%%%%%%%%%%%%%%%%
%% Theorem definitions
%%%%%%%%%%%%%%%%%%%

\usepackage{aliascnt}

\newcommand{\thdef}[2]{
	\newaliascnt{#1}{theorem}  
	\newtheorem{#1}[#1]{#2}
	\aliascntresetthe{#1}  
	\newtheorem*{#1*}{#2}
	\expandafter\newcommand\expandafter{\csname #1autorefname\endcsname}{#2}
}


\newtheorem{theorem}{Theorem}[subsection]
\newtheorem*{theorem*}{Theorem}
\thdef{conjecture}{Conjecture}
\thdef{problem}{Problem}
\thdef{lemma}{Lemma}
\thdef{corollary}{Corollary}
\thdef{proposition}{Proposition}


\theoremstyle{definition}
\thdef{definition}{Definition}
\thdef{notation}{Notation}
\thdef{question}{Question}
\thdef{remark}{Remark}
\thdef{example}{Example}
\thdef{construction}{Construction}
\thdef{caution}{Caution}
\thdef{exercise}{Exercise}
\thdef{fact}{Fact}

\AtEndEnvironment{fact}{%
  \ifqedbarused\else\qedbar\fi%
  \global\qedbarusedfalse% 
}
\AtEndEnvironment{remark}{%
  \ifqedbarused\else\qedbar\fi%
  \global\qedbarusedfalse% 
}
\AtEndEnvironment{definition}{%
  \ifqedbarused\else\qedbar\fi%
  \global\qedbarusedfalse%
}
\AtEndEnvironment{example}{%
  \ifqedbarused\else\qedbar\fi%
  \global\qedbarusedfalse%
}
\AtEndEnvironment{notation}{%
  \ifqedbarused\else\qedbar\fi%
  \global\qedbarusedfalse%
}
\AtEndEnvironment{caution}{%
  \ifqedbarused\else\qedbar\fi%
  \global\qedbarusedfalse%
}
\AtEndEnvironment{construction}{%
  \ifqedbarused\else\qedbar\fi%
  \global\qedbarusedfalse%
}
\AtEndEnvironment{exercise}{%
  \ifqedbarused\else\qedbar\fi%
  \global\qedbarusedfalse%
}
\numberwithin{equation}{section}

%%%%%%%%%%%%%%%%%%%%%%%%%%%%%
%%%%%%%%%% Titre %%%%%%%%%%%%
%%%%%%%%%%%%%%%%%%%%%%%%%%%%%

\title{Hyperkähler manifolds}
\author{Notes taken by Maxim Jean-Louis Brais}
\date{\today}

%%%%%%%%%%%%%%%%%%%%%%%%%%%%%%%%%%%%%%%%%%%%
%% Counter for table of contents
%%%%%%%%%%%%%%%%%%%%%%%%%%%%%%%%%%%%%%%%%%%%

\setcounter{tocdepth}{1}


\begin{document}
\maketitle
\section{First lecture}
We first review some complex geometry.
\begin{definition}
	A \tbf{complex manifold} is a locally ringed space $(X,\cO_X)$ such that
	\begin{itemize}
		\item $X$ is Hausdorff and second countable (this part is to ensure we actually have a topological manifold);
		\item $(X,\cO_X)$ is locally isomorphic to $(\Delta,\cO_\Delta)$, where $\Delta\subset \C^n$ is the polydisc.\qedbarhere 
	\end{itemize} 
\end{definition}
\begin{example}\label{eg: affine var}
	Let $f_i,\dots,f_d\in\C[z_1,\dots,z_n]$ be complex polynomials such that the Jacobian of
	\[f=(f_1,\dots,f_d):\C^n\to \C^d
	\]
	has everywhere full rank on the vanishing set $V=V(f)\subset \C^n$. By the holomorphic implicit function theorem (regular value theorem), $V$ is a complex manifold.
\end{example}
\begin{example}\label{eg: smooth algebraic var}
	If $X$ is a smooth algebraic variety over $\C$, we may cover it by affines $V_i$ which are of the same form as in \autoref{eg: affine var}. We may consider the analytic topology $X^{an}$ obtained by gluing the different charts $V_i$. Similarly, we may define the sheaf $\cO_X^{an}$ on $X^{an}$ by considering the sheaf of holomorphic functions on each $V_i$ (the transitions $V_i\to V_j$ are regular algebraic, hence holomorphic, so that this gluing makes sense). Then, $(X^{an},\cO_X^{an})$ is a complex manifold.
\end{example}

Note that in \autoref{eg: smooth algebraic var}, we obtain a natural map of ringed spaces
%
\[
\alpha:(X^{an},\cO_X^{an})\to (X,\cO_X)
\]
since the analytic topology is finer than the Zariski topology, and regular functions are holomorphic. In particular, we obtain a functor between abelian categories:
\[
\alpha^*:\cO_X\text{-mod}\to \cO_X^{an}\text{-mod}
\]
restricting to
\[
\alpha^*:\text{Coh}(X)\to \text{Coh}(X^{an}).
\]
\begin{theorem}[Géométrie algébrique géométrie analytique;\;\cite{Serre1956}]
If $X$ is smooth\footnote{This can be dropped by considering complex analytic spaces (rather than manifolds).} and proper functor $\alpha^*:\text{Coh}(X)\to\text{Coh}(X^{an})$ is an equivalence, therefore inducing an isomorphism.
\end{theorem}
\subsection{Almost complex structures}
A complex manifold $(X,\cO_X)$ has an underlying smooth manifold $(X, C^\infty_X)$, where $C^\infty_X$ denotes the sheaf of smooth functions on $X$; indeed, if $X$ has complex charts $U_i\subset \C^n$, the transitions are holomorphic, hence $C^\infty$. 
\begin{notation}
	Since the indices $i$ will be ubiquitous, $\iota$ shall denote the root of $-1$ for these notes (this spares the cumbersome $\sqrt{-1}$ alternative).
\end{notation}
On each chart $U_i$, we have multiplication by $\iota$, but this does not globalise, as $\iota$ does not commute with holomorphic functions: in the Taylor expansion, we have terms which are of degree $m$ where $m\neq 1\mod 4$. However, the differential of $\iota$ may be globalised, as we get rid of the higher order terms. In a local chart $U_i\subset \C^n$, the (real) tangent bundle has a local frame
\[
T_\R U_i=\langle \partial_{x_j},\partial_{y_j}:1\leq j\leq n\rangle,
\]
on which $I:=d\iota$ acts by
\[
\begin{cases}
	\partial_{x_j}\mapsto \partial_{y_j}\\
	\partial_{y_j}\mapsto -\partial_{x_j}.
\end{cases}
\]
\begin{definition}
	An \tbf{almost complex structure} on a smooth manifold $X$ is an endomorphism $I\in \text{End}(T_\R X)$ such that $I^2=-1$. We say that $I$ is integrable if $X$ is a complex manifold and $I$ is obtained by locally differentiating $\iota$. 
\end{definition}
\begin{question}\label{q: integrable?}
	Given $I$ an almost complex structure, when is it integrable?
\end{question}
Let us first set up some tools in order to address this question appropriately. Assume only for now that $X$ is a smooth manifold and $I$ is an almost complex structure. We can consider the complexified tangent bundle 
\[
T_\C X:=T_\R X\otimes_\R \C,
\]
to which we can extend the action of $I$. Since $I^2={-1}$, the minimal polynomial of $I$ is $x^2+1$, which is separable over $\C$, meaning that $I$ is diagonalisable, with eigenvalues $\pm \iota$. The eigenspaces must have the same dimension as $I$ acts on the \emph{real} tangent space. We thus obtain a decomposition
\[
T_\C X=T^{1,0}X\oplus T^{0,1}X= T^{1,0}X\oplus \overline{T^{1,0}X}.
\]
Note that we have
\begin{align*}
	T^{1,0}X=\{(v-iIv):v\in T_\C X\}&&T^{0,1}X=\{(v+iIv):v\in T_\C X\}.
\end{align*}
\begin{notation}
	We will use the following notation
	\begin{itemize}
		\item $\cA^0(X):=C^\infty_X$;
		\item $\cA^k(X)$ denotes the sheaf of (smooth) degree $k$ real forms;
		\item $\cA^k(X,\C)=\oplus_{p+q=k}\cA^{p,q}(X)$ denotes the sheaf of sections of $\bigwedge^kT_\C^*X$ (i.e. smooth complex degree $k$ forms) and $\cA^{p,q}(X)$ denotes the sheaf of sections of $\bigwedge^pT^{1,0}X\otimes \bigwedge^q T^{0,1}X$;
		\item $d:\cA^k(X,\C)\to \cA^{k+1}(X,\C)$ denotes the complexification of the usual exterior derivative, and can be decomposed by types as $d=\partial+\bar\partial$, where $\partial$ denotes the part corresponding to the differentiation in holomorphic coordinates, and similarly $\bar\partial$ for anti-holomorphic coordinates.
		\item $A^k(X)$, $A^k(X,\C)$, and $A^{p,q}(X)$ denotes the global sections of respectively $\cA^k(X),\cA^k(X,\C)$, and $\cA^{p,q}(X)$.\
		\item $\cT_X$ denotes the sheaf of homolorphic vector fields, i.e. $\cT_X:=\cD er(\cO_X,\cO_X)$;
		\item $\Omega_X:=\cT_X^*$ denotes the cotangent sheaf.\qedbarhere
		\end{itemize}
\end{notation}
The following theorem answers \autoref{q: integrable?}.
\begin{theorem}[Newlander-Niremberg]
	$I$ is integrable if and only if $\bar{\partial}^2=0$.
\end{theorem}
Note that this is equivalent to $T^{1,0}X$ being closed under the (complex) Lie bracket (this is much related to the Frobenius theorem of differential geometry), and also equivalent of the vanishing of a certain tensor $N_I$ called the \emph{Nijenhuis} tensor. 
\subsection{Metrics}
Let $E$ be a real vector bundle on $(X,C_X^\infty)$. A \tbf{Riemannian metric} $g$ on $E$ is a section of $\text{Sym}^2E^\vee$ such that for all $p\in X$, $g_p$ is positive definite. If $E$ is a complex bundle, a \tbf{Hermitian metric} $h$ is a map of sheaves $E\otimes \overline E\to C^\infty(X,\C)$ such that each $h_p$ is Hermitian, i.e. $h_p(e,f)=\overline{h_p(f,e)}$ and $h_p(e,e)>0$ for all $e,f\in E_p$.  When $E=T_\R X$, we say that $g$ (resp. $h$) is a \tbf{Riemannian} (resp. \tbf{Hermitiam}) \tbf{metric on } $X$ (here, the almost complex structure $I$ is used to put a $\C$-structure on $T_\R X$).

If $X$ is a complex manifold and $h$ is a  Hermitian metric, then we can write
\[
h=g-i\omega
\]
where $g=\fR\fe(h)$ and $\omega=-\fI\fm(h)$. We obtain that $g$ is a Riemannian metric, and $\omega$ is skew-symmetric since
\[
\omega(X,Y)=\frac{\iota}{2}(h-\overline h)
\]
and $h$ is conjugate skew-symmetric. Thus, $\omega\in A^2(X)$.
\begin{definition}
	$(X,h)$ is \tbf{Kähler} if $d\omega=0$.
\end{definition}
That $h$ is linear in the first variable and anti-linear in the second ensures that $h(I-,I)=h(-,-)$, implying that $g(I-,I-)=g(-,-)$, a property that is sometimes called \tbf{compatibility} of the metric with $I$. We have 
\[
\omega(-,-)=\frac{\iota}{2}(h(-,-)-\overline h(-,-))=\frac{1}{2}(h(I-,-)+\overline h (I-,-))=g(I-,-),
\]
which also implies
\[
\omega(-,I-)=g(-,-).
\]
\begin{definition}\label{def: kahler}
	A form $\omega\in A^2(X)$ is called \tbf{positive} if $\omega(u,Iu)>0$ for all $u\in T_\R X$. We see that a de Rham cohomology class in $H^2(X,\C)$ is \tbf{positive} if it can be represented by a positive form. If moreover $\omega$ is $I$-invariant (or equivalently, of type $(1,1)$ after embedding $A^2(X)\subset A(X,\C)$), we say $\omega$ is \tbf{Kähler}.
\end{definition}
If $\omega$ is Kähler, we may define the hermitian metric $h_\omega=\omega(-,I-)-i\omega$, and we have that $\omega$ is Kähler if and only if $(X,h_\omega)$ is Kähler.
\begin{example}
	Let $X=\bP^n$, with projective coordinates $Z_0,\dots,Z_n$. Let $U_i$ be the $Z_i\neq 0$ chart, and define $z_j=\frac{Z_j}{Z_i}$. We may define on $U_i$ the metric
	\[
	\omega_{FS}=\omega=i\partial\overline\partial\log\pa{1+\sum_jz_j\overline z_j},
	\]
	and one checks that these glue to a global form, which we call the \tbf{Fubini-Study metric}. Written as a Kähler potential this way shows that it is a Kähler metric.
\end{example}
Note that if $(X,\omega)$ is Kähler, restricting the metric to a complex submanifold $Y$ preserves all properties of \autoref{def: kahler}, and so $(Y,\omega_Y)$ is Kähler. Thus, any projective manifold is Kähler.

\subsection{Connections}

Let $E$ be a complex (the real case is identical) vector bundle on $(X,C_X^\infty)$. A \tbf{complex connection} in $E$ is a $\C$-linear map 
\[
\nabla:\cA^0(E,\C)\to \cA^1(E,\C),
\]
(here $\cA^i(E,\C)= \cA^i(X,\C)\otimes \Gamma(E)$) such that
%
\[
\nabla(f\cdot s)=df\otimes s+f\cdot\nabla s
\]
for all section $s$ of $E$ and $f\in C^\infty_X$.

If $E$ is a holomorphic bundle on a complex manifold, we can define the operator
%
\[
\overline\partial:\cA^0(E)\to \cA^{0,1}(E)
\]
%
as follows: if $\sigma_i$ is a local frame, and $s=s^i \sigma_i$ a section, we let
\begin{equation}
\overline\partial(s^i\sigma_i):=(\overline\partial s^i)\otimes \sigma_i.\label{eq: dbar bundle}
\end{equation}

Indeed, given another frame $\tau_j$ related by $\sigma_i=g_{ij}\tau_j$, we have
\[
\overline\partial (s^i)\otimes \sigma_i=\overline \partial(s_i)\otimes g_{ij}\tau_j=\overline\partial(g_{ij}s^i)\otimes \tau_j
\]
since the transitions $g_{ij}$ are holomorphic by assumption.

A (complex) connection being valued in $\cA^1(E,\C)=\cA^{1,0}(E)\oplus\cA^{0,1}(E)$, we may split $\nabla=\nabla^{1,0}+\nabla^{0,1}$.
\begin{definition}
	The complex connection $\nabla$ in $E$ is said to be \tbf{compatible} with the holomorphic structure if $\nabla^{0,1}=\overline\partial$. Suppose $E$ has a hermitian metric $h$. We say $\nabla$ is \tbf{compatible} with $h$ if for any sections $e,f$, we have equality of forms
	\[d(h(e,f))=h(\nabla e,f)+h(e,\nabla f).\]
More geometrically, this says that $h$ is parallel to the connection, i.e. constant along parallel transport, i.e. the connection has $U(n)$-holonomy. We say $\nabla$ is a \tbf{Chern connection} if it is both compatible with the holomorphic structure and the hermitian metric.
\end{definition} 

\begin{theorem}[Chern]
	There exists a unique Chern connection.
\end{theorem}

When $E=T_\R X$, the Chern connection ought to be regarded as the complex geometric analogue of the Levi-Civita connection from Riemannian geoemtry. In fact this is more than an analogy. If $h$ is a hermitian metric, the Levi-Civita connection of $g=\fR\fe(h)$ can be complexified to a complex connection. It is a theorem that the Levi-Civita connection is the Chern connection if and only if $(X,h)$ is Kähler.

We can extend the connection $\nabla:\cA^0(E,\C)\to \cA^1(E,\C)$ to a connection
\[
\nabla:\cA^p(E,\C)\to \cA^{p+1}(E,\C)
\]
for all positive $p$ via
\[
\nabla(\omega\otimes s)=d\omega\otimes s+(-1)^p\omega\wedge\nabla s,
\]
where $\omega$ is a $p$-form and $s$ is a section of $E$.

\begin{remark}
	Note this different to the usual extension of a connection to tensors since we are dealing with skew-symmetric forms. In particular, this satisfied a different Leibniz rule:
	\[
	\nabla(fs)=df\wedge s\otimes d\nabla s.
	\]
\end{remark}

\begin{definition}
	We define the \tbf{curvature} of $\nabla$ to be the composition $\nabla^2=\nabla\circ\nabla=F_\nabla$.
\end{definition}

Note that
\begin{align*}
 \nabla(\nabla fs)=\nabla(df\otimes s+f\nabla s)&=ddf-df\wedge\nabla s+\nabla(f\nabla s)	\\
 &=-df\wedge\nabla s+df\wedge\nabla s+f\nabla^2s=f\nabla^2s
\end{align*}
so that $F_\nabla$ is $C^\infty_X$-linear, that is a section of $\cA^2(End(E),\C)$.

We may also define
%
\[F_\nabla^k:=\underbrace{F_\nabla\circ\cdots\circ F_\nabla}_k\in \cA^{2k}(End(E),\C).
\]
We define the \tbf{$k$th Chern character} of $\nabla$ to be
%
\[
\text{ch}_k(E,\nabla):= \Tr\pa{\frac{1}{k!}\pa{\frac{\iota}{2\pi}F_\nabla^k}}\in A^{2k}(X,\C).
\]

\begin{theorem}[Chern-Weil]
The following is true about the Chern character.
\begin{enumerate}
	\item $\text{ch}_k(E,\nabla)$ is closed;
	\item The cohomology class $\text{ch}_k(E):=[\text{ch}_k(E,\nabla)]\in H^{2k}_{dR}(X,\C)$ is independent of $\nabla$;
	\item $\text{ch}_k(E)$ is real, i.e. in $H^{2k}_{dR}(X,\R)$ (in fact, it is integral);
	\item The total Chern character $\sum_k\text{ch}_k(E)$ is equal to the cohomology class of $\Tr(\exp(\frac{\iota}{2\pi}F_\nabla)$ (this one directly follows from developing the exponential).
\end{enumerate}	
\end{theorem}

\section{Second Lecture}
\subsection{Chern classes}
Let $V$ be a vector space over $\C$ of dimension $r$. Let $P\in \C[End(V)]$ be a homogeneous polynomial of degree $k$. Assume moreover $P$ is $GL(V)$ invariant, that is $P(A^{-1}BA)=P(B)$ for any $A\in GL(V)$.

Let now $E$ be a complex vector bundle and $\nabla$ a connection. By $GL(V)$ invariance, $P(\frac{\iota}{2\pi}F_\nabla)$ is well-defined, and lives in $A^{2k}(X,\C)$.
\begin{fact}[Chern-Weil]
	$P(\frac{\iota}{2\pi}F_\nabla)$ is closed, and the class $[P(\frac{\iota}{2\pi}F_\nabla)]\in H^{2k}(X,\C)$ is independent of $\nabla$.
\end{fact}
Consider now the $GL(V)$-invariant homogeneous polynomials $P_k$ returning the coefficients of the characteristic polynomials (i.e. $P_k$ $k$th elementary symmetric polynomial on the eigenvalues). We can explicitely define $P_k$ by the formula:
\[
\det(I+tB)=\sum_kP_k(B)t^k.
\]

We define the \tbf{$k$th Chern class} of $E$ to be the cohomology class of $P_k(\frac{i}{2\pi}F_\nabla)$, and the \tbf{total Chern class} of $E$ to be $c(E):=\sum_{i=0}^kc_k(E)$.

The Chern classes and characters satisfy certain properties:
\begin{itemize}
	\item $c_0(E)=1$ and $ch_0=r$;
	\item $c_d=0$ if $d>r$. In particular, if $L$ is a line bundle, $c(L)=1+c_1(L)$;
	\item $ch(L)=\exp(c_1(L)):=1+c_1(L)+\frac{c_1(L)^2}{2!}+\cdots\in H^{2\bullet}(X,\C)$.
\end{itemize}

To derive further elementary properties of the Chern characters and classes, let us first observe how we can assemble connections into new connections

	Let $E_1$ and $E_2$ be vector bundles with respective (complex) connections $\nabla_1$ and $\nabla_2$. Then,
	\begin{itemize}
		\item $\nabla_{E_1\oplus E_2}:=\nabla_1\oplus\nabla_2$ is a connection on $E_1\oplus E_2$, and $F_{\nabla_1\oplus\nabla_2}=F_{\nabla_1}\oplus F_{\nabla_2}$, where this is seen as a block matrix
		\[\begin{pmatrix}
			F_{\nabla_1}&\\&F_{\nabla_2}
		\end{pmatrix}; \]
		\item $\nabla_{E_1\otimes E_2}:=\nabla_1\otimes \id_{E_2}+\id_{E_1}\otimes \nabla_2$ is a connection on $E_1\otimes E_2$;
		\item the assignment
		\[
		(\nabla^\vee \phi)(s):=d(\phi(s))-\phi(\nabla(s))
		\]
		where $s\in E$ and $\phi\in E^\vee$ defines a connection on $E^\vee$ (note that this is the usual way to extend connections on tensors). In other words, if $\langle-,-\rangle$ denotes the natural pairing on $E\otimes E^\vee$, the dual connection is defined by
		\[
		d\langle s,\phi\rangle=\langle\nabla s,\phi\rangle+\langle s,\nabla^\vee \phi\rangle.
		\]
		Let us try to compare the two curvature. Given $s_i$ ant $t_j$ frames of $E$, we consider the connexion form $A=(A_i^{\;j})$ satisfying $\nabla s_i=A_i^{\;j}\otimes t_j$. Let $s^i$ and $t^j$ be the dual frames. We obtain
		\begin{align*}
			d\langle s_i,t^j\rangle=0&=\langle \nabla s_i,t^j\rangle+\langle s_i,\nabla^\vee t^j\rangle\\
			&=\langle A_i^{\;k}\otimes t_k,t^j\rangle+\langle s_i,B^j_{\;k}\otimes s^k\rangle\\
			&=A_i^{\;j}+B^j_{\; i},
		\end{align*}
		where $B=(B^j_{\; i})$ is the connection form of $\nabla^\vee$. And so we have $B=-A^t$ as sections of $\cA^2(End(E),\C)=\cA^2(End(E^\vee),\C)$. Using Cartan's formula for the curvature of a connection, we conclude
		\[
		F_{\nabla^\vee}=d(-A^t)+(-A^t)\wedge(-A^t)=-(dA+A\wedge A)^t=-F_{\nabla}^t.
		\]
		\item connections pull back, that is if $f:Y\to X$ is a smooth map and $E$ is a bundle on $X$ with connection $\nabla_E$, we may define the connection $\nabla_{f^*E}$ by \emph{locally} demanding
		\[
		\nabla_{f^*E}(f^*s)=f^*\nabla s.
		\]
	\end{itemize}


\begin{corollary}
	Let $E_1,E_2$ be complex vector bundles on $X$. The following hold:
	\begin{enumerate}
		\item $ch(E_1\oplus E_2))=ch(E_1)+ch(E_2)$ and $c(E_1\oplus E_2)=c(E_1)\cup c(E_2)$;
		\item $ch(E_1\otimes E_2)=ch(E_1)\cup ch(E_2)$;
		\item $c_k(E^\vee)=(-1)^k c_k(E)$
		\item $ch_k(f^*E)=f^*(ch_k(E))$ and  and $c_k(f^*E)=f^*(c_k(E))$.
	\end{enumerate}
\end{corollary}

Note that $c_k(E)\in H^{2k}(X,\C)$ is in fact real: indeed, conjugation acts on via $\overline{F_\nabla}=F_{\overline\nabla}$, and up to choosing a hermitian metric, we have $E^\vee\simeq \overline E$ and $F_{\overline\nabla}=-F_\nabla^t$, and in a Chern splitting, the eigenvalues $\omega_1,\dots,\omega_r$ are purely imaginary. We obtain
\[c_k(E)=P_k(\frac{\iota}{2\pi}F_\nabla)=\brk{\frac{\iota^k}{(2\pi)^k}P_k(F_\nabla)}=\brk{\frac{\iota^k}{(2\pi)^k}\sigma_k(\omega_1,\dots,\omega_r)}
\]
while
\[
\overline{c_k(E)}=\brk{\frac{(-1)^k\iota ^k}{(2\pi)^k}\sigma_k(-\omega_1,\dots,-\omega_r)}=\brk{\frac{(-1)^{2k}\iota^k}{2\pi}\sigma_k(\omega_1,\dots,\omega_r)}=c_k(E)
\]
where $\sigma_k$ denotes the $k$th standard symmetric polynomial, so that $c_k(E)\in H^{2k}(X,\R)$. The $k$th Chern class is also $(1,1)$. Indeed, consider the Chern connection $\nabla=\nabla^{1,0}+\nabla^{0,1}=\nabla^{1,0}+\overline\partial$. From this decomposition, we see that the $(0,2)$-part of the curvature is $\overline\partial^2=0$. Similarly, one can use the fact that the hermitian metric is parallel to show that the $(2,0)$ part vanishes so that all $\omega_i$ are of type $(1,1)$, from which one obtains that $c_k(E)$ is of type $(k,k)$.

Let $D\subset X$ be a divisor. It is a fact that the fundamental class of $D$ is the Chern class of $\cO(D)$, i.e.
\[
[D]=c_1(\cO(D))\in H^{2}(X,\C),
\]
showing that the first\,---\,and therefore fore any\,---\,Chern class is integral.

An important theorem relating to chern classes is Kodaira's embedding theorem.
\begin{theorem}[Kodaira]
	A (holomorphic) line bundle $L$ on a complex manifold $X$ is ample (i.e. induces an embedding in projective space) if and only if it admits a metric $h$ such that $\frac{\iota}{2\pi}F_{D_h}$ is a positive form.
\end{theorem}
In particular, let $h_0$ be any hermitian metric on $L$, and let $\omega_0=\frac{\iota}{2\pi}F_{D_{h_0}}$. Assuming $X$ is Kähler, if $c_1(L)=[\omega_0]$ is positive, i.e. if it has a positive form $\omega$ as representative of the cohomology class, then we can write $\omega=\omega_0+\frac{\iota}{2\pi}\partial\overline\partial\phi$ for some function $\phi$ by the $\partial\overline\partial$-lemma. Then, one may compute that the metric $h:=e^{-\phi}\cdot h_0$ satisfies $\frac{\iota}{2\pi}F_{D_h}=\omega$; indeed the $(1,0)$-part of the Chern of the connection is
\[
h^{-1}\partial h=\partial\log h=\partial\log (e^{-\phi}h_0)=\partial\log h_0+\partial (-\phi)
\]
so that the curvature is given by
\[F_{D_{h_0}}+\overline\partial\partial(-\phi)=F_{D_{h_0}}+\partial\overline\partial\phi.
\]
 In particular, a line bundle $L$ is ample if and only if $c_1(L)$ is positive, i.e. is represented by a Kähler form. Now, on a compact manifold, slightly perturbing a Kähler form inside $H^{1,1}(X,\R)$ still yields a Kähler form, since it preserves the positivity criterion. Thus, Kähler forms form an open positive cone $\cK_X$ inside of $H^{1,1}(X,\R)$ (scaling by a positive real preserves Kählerness). Moreover, by the Lefschetz theorem on $(1,1)$ classes, the map Chern map $\text{Pic}(X) \to H^{1,1}(X,\Z)$ is surjective.  Thus, we conclude that a compact complex manifold is projective if and only if $\cK_X$ intersects with $H^{1,1}(X,\Z)$ inside of $H^{1,1}(X,\R)$.
\subsection{Hirzebruch-Riemann-Roch}
\begin{definition}
	Let $X$ be a compact complex manifold. Let $\nabla$ be a connection in the tangent bundle. We define the \tbf{Todd class} of $X$ to be
	\[
	td(X):=\brk{\det\pa{\frac{\frac{\iota}{2\pi}F_\nabla}{1-\exp(\frac{-\iota}{2\pi}F_\nabla}}}\in H^{2\bullet}(X,\C)
	\]
\end{definition}
In terms of Chern roots $\omega_1,\dots,\omega_r$, we have that
\[
td(X)=\prod_{i=1}^r\frac{\omega_i}{1-e^{-{\omega_i}}}\]
It can be computed that we have
\begin{align*}
	td_0(X)=1;&&td_1(X)=\frac{c_1}{2};&&td_2(X)=\frac{1}{12}(c_1^2+c_2);&& td_3(X)=\frac{c_1c_2}{24}&&td_4(X);=\frac{-c_1^4+4c_2+c_1c_3+3c_2^2-c_4}{720};&&\cdots
\end{align*}

where $td_k(X)$ denotes the $k$th homogeneous component of $td(X)$ and $c_i=c_i(X):=c_i(\cT_X)$.
\begin{theorem}[Hirzebruch-Rieman-Roch]
	Let $E$ be a holomorphic vector bundle on a compact complex manifold $X$. Then, we have equality
	\begin{equation}
	\chi(X,E):=\sum_k(-1)^k h^i(X,E)=\int_X ch(E)\cup td(X).\label{eq: HRR}
	\end{equation}
\end{theorem}
Note that since we are integrating over $X$, we only need to consider the top degree parts.
\begin{example}\label{eg: HRR curves}
	Let $X=C$ be a compact Riemann surface and $L$ be a line bundle on $X$. we have $ch(E)=1+c_1(L)$ and $td(1)=1+\frac{c_1(X)}{2}$. thus, we have
	\[
	\chi(X,L)=\int_X 1+c_1(L)+\frac{c_1(X)}{2}+\frac{c_1(L)c_1(X)}{2}=\int_X c_1(L)+\frac{c_1(X)}{2}=\deg(L)+\frac{\deg(\cT_C)}{2}.
	\]
\end{example}
What is remarkable about this theorem is that the left-hand side of \eqref{eq: HRR} is purely holomorphic (or algebraic) whilst the right-hand side is purely topological. Another similar theorem is the algebro-geometric Gauss-Bonnet theorem.
\begin{theorem}
	Let $X$ be a compact complex dimension of dimension $n$. Then, 
	\[
	\chi_{top}(X):=\sum_i(-1)^i b_i(X)=\int_Xc_n(X).
	\]
\end{theorem}

Recall the classical relation between the Euler characteristic $e$ and the genus $g$ of a topological surface: $\chi_{top}=2-2g$. In particular, this implies that for a compact Riemann surface $C$ as above, that
 \[
	\int_Xc_1(X)=\deg(\cT_C)=2-2g.
	\]
In particular, in light of what we found in \autoref{eg: HRR curves}, we recover the classical Riemann-Roch theorem:
\[
\chi(X,L)=\deg(L)-g+1.
\]
\subsection{Kähler-Einstein manifolds}
\begin{question}
	When does a smooth projective variety over $\C$ admit a ``canonical'' metric?
\end{question}
\begin{definition}
	Let $(X,\omega)$ be a compact Kähler manifold, and let $D_\omega$ be the corresponding Chern connection. We define the \tbf{Ricci form} $\text{Ric}(\omega)$ of $\omega$ to be
	\[
	\text{Ric}(\omega)=i\Tr(F_{D_\omega})\in A^2(X,\C).
	\]
	We say that $(X,\omega)$ is \tbf{Kähler-Einstein} if $\text{Ric}(\omega)=\lambda\omega$ for some constant $\lambda\in \R$. 
\end{definition}
\begin{remark} We make the following comments.
\begin{enumerate}
	\item  Recall that we argued earlier that all the Chern roots of $F_{D_\omega}$ were pure imaginary of type $(1,1)$ so that $\text{Ric}(\omega)$ is real of type $(1,1)$. 
	
	\item Note also that $D_\omega$ is invariant under rescaling $\omega$ by some $\lambda>0$ (indeed, parallelness of $h$ is unaffected so we get the same connection). Thus, we may always assume that $\lambda=-1,0,1$.
	
	\item since $c_1(X)=[\frac{\iota}{2\pi}\Tr F_{D_\omega}]$ by definition, we have $[\text{Ric}(\omega)]=2\pi c_1(X)\in H^2(X,\R)$.
	\item $\lambda$ is proportional to the scalar curvature, and so $(X,\omega)$ being Kähler-Einstain implies that the scalar curvature with respect to $g_\omega=\omega(I-,-)$ is constant.
	\item If $X$ is Kähler-Einstein, then we have
	\[
	c_1(X)=\begin{cases}
		0\\
		\pm\text{ positive form}.
	\end{cases}
	\]
\end{enumerate}	
\end{remark}
\begin{definition}
	We say that a complex manifold $X$ is \tbf{Calabi-Yau} if $c_1(X)=0$, \tbf{Fano} if $c_1(X)$ is positive, of \tbf{general type} (or \tbf{canonically polarised}) if $-c_1(X)$ is positive.
\end{definition}
Note that by Kodaira's embedding theorem, Fano and general type manifolds are projective.
\begin{caution}
	It is not because a manifold fits in this trichotomy that it admits a Kähler-Einstein metric. In fact, there exist Fano varieties with no Kähler-Einstein metric. Whether a Fano variety admits such metric is equivalent to $K$-stability, a purely algebro-geometric notion. Nonetheless, Yau (cf. \cite{Yau1978}) proved that any Calabi-Yau manifodl admits a Kähler-Einstein metric, and Aubin--Yau (cf. \cite{Aubin78,Yau1978}) proved the same for general type manifolds.
	
	Note also that not all manifolds fit in this trichotomy. 
\end{caution}
\begin{example}[Curves]
Let us see how these categories apply to curves.
\begin{itemize}
	\item $g=0$ gives only $\bP^1$. Since it is diffeomorphic to a sphere, we have positive scalar curvature. And indeed, the Fubini-Study metric is Kähler-Einstein with $\lambda=1$. Note also that $\bP^1$ is Fano.
	\item The $g=1$ case corresponds to elliptic curves. These are Ricci-flat and Calabi-Yau.
	\item The case $g>1$ are of general type, and there exists a Kähler-Einstein metric with negative scalar curvature.\qedbarhere
\end{itemize}
\end{example}
For Fano manifolds, here is a summary of the known classifications:
\begin{enumerate}
	\item In dimension $1$ there is only the projective line.
	\item In dimension $2$, they are called \emph{del Pezzo} surfaces. There are $10$ different deformation families. First, $\bP^2$ and $\bP^1\times\bP^1$ which are isolates. The other $8$ families are obtained by blowing up $\bP^2$ at d points in general position, where $1\leq d\leq 8$.
	\item In dimension $3$, Mukai proved there are 105 families.
	\item In dimension $4$, we know there are finitely many families but it remains open to know how many.
\end{enumerate}

For Calabi-Yau manifolds, there is the following structural theorem.
\begin{theorem}[Beauville--Bogomolov]
	Let $X$ be Kähler and Calabi-Yau. Then, there exists an étale cover $\tilde X\to X$ such that 
	\[
	\tilde X=T\times \prod_jX_j\times \prod_i V_i
	\]
	where $T$ is a torus, $X_j$ is \tbf{hyperkähler} for all $j$, and $V_i$ are \tbf{strict Calabi-Yau} for all $i$. 
\end{theorem}
We now define the terms.
\begin{definition}
	A compact Kähler manifold $V$ is called \tbf{strict Calabi-Yau} if
	\begin{itemize}
		\item $K_V\simeq \cO_V$ is trivial, where $K_V$ denotes the canonical bundle;
		\item $V$ is simply connected;
		\item $H^i(V,\cO_V)=0$ for all $i<i<\dim V$.
	\end{itemize}
	A complex manifold $X$ is \tbf{hyperkähler} if
	\begin{itemize}
		\item it is simply connected;
		\item $H^0(X,\Omega_X^2)\simeq \C\sigma$ where $\sigma$ is holomorphic symplectic (in particular, it induces an isomorphism $\cT_X\simeq \Omega_X$).\qedbarhere
	\end{itemize}
\end{definition}
\begin{remark}
	If $V$ is a strict Calabi-Yau of dimension greater than two, then $h^{2,0}=h^{0,2}=0$. In particular, $H^{1,1}(X,\C)=H^2(X,\C)$, and so $H^{1,1}(X,\R)=H^{2}(X,\R)=H^{1,1}(X,\Z)\otimes_\Z\R$. Since the Kähler cone is not empty by assumption, we conclude that it intersects $H^{1,1}(X,\Z)$, so that $V$ is projective by our discussion on Kodaira's embedding theorem. In dimension $2$, non-projective K3 surfaces yield an example of a non-projective strict Calabi-Yau manifold.
\end{remark}
\printbibliography
\end{document}

